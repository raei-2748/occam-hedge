\documentclass{article}

% Encoding and typography
\usepackage[utf8]{inputenc}
\usepackage[T1]{fontenc}
\usepackage{textcomp}
\usepackage{microtype}


% Math
\usepackage{amsmath,amssymb,amsthm}
\DeclareMathOperator{\sgn}{sgn}

% Figures and tables
\usepackage{graphicx}
\usepackage{booktabs}
\usepackage{longtable}
\usepackage{caption}
\usepackage{tabularx}
\usepackage{float}
\usepackage{array}
\usepackage{placeins}

% Bibliography
\usepackage[round,authoryear]{natbib}

% Hyperlinks (load late)
\usepackage{hyperref}
\hypersetup{
 colorlinks=true,
 linkcolor=blue,
 citecolor=blue,
 urlcolor=blue
}
\usepackage[nameinlink,capitalize]{cleveref}

% Theorems
\theoremstyle{definition}
\newtheorem{definition}{Definition}[section]
\newtheorem{assumption}{Assumption}[section]

\theoremstyle{plain}
\newtheorem{lemma}{Lemma}[section]
\newtheorem{corollary}{Corollary}[section]

\newtheoremstyle{uprightprop}
 {6pt}{6pt} % Space above/below
 {\normalfont} % Body font
 {} % Indent amount
 {\bfseries} % Head font
 {.} % Punctuation after head
 {0.5em} % Space after head
 {} % Head spec
\theoremstyle{uprightprop}
\newtheorem{proposition}{Proposition}[section]


\title{Occam's Hedge under Relative Entropy Uncertainty}
\author{Ray Wang\\\texttt{raywang886@gmail.com}}
\date{}

\begin{document}
\maketitle


\begin{abstract}
Hedging policies that exploit market microstructure signals face a hidden fragility. When intermediation constraints bind, the relationship between activity proxies and execution costs can invert sign, causing identical observations to imply opposite optimal responses across regimes. We formalize this as \emph{semantic inversion} and show it leads to wrong-way trading, where policies trade aggressively precisely when costs are highest. We prove that safely using sign-unstable signals requires encoding regime-distinguishing information, creating a fundamental trade-off between baseline efficiency and stress robustness. We propose \emph{Occam's Hedge}, which prices representational bandwidth through a variational information bottleneck, implementing structural priors that distinguish payoff-anchored exposures from equilibrium-derived proxies. In a controlled identification environment where regime information is unavailable from single snapshots by design, all representations exhibit near-identical stress degradation (Deg $\approx 1.16$), confirming that microstructure cannot be safely exploited without regime disambiguation. As information constraints tighten, policies shed dependence on semantically unstable signals: wrong-way exposure drops from $+0.11$ to near zero and the degradation ratio falls from 1.16 to 1.0, though baseline risk increases by 60\%. An oracle test providing regime labels eliminates degradation entirely (Deg $= 0.97$), confirming the failure is purely informational.
\end{abstract}


\noindent\textbf{Keywords:} deep hedging; execution costs; regime shifts; information bottleneck; market microstructure.\\
\textbf{JEL Codes:} G13, C61, G11.

\section{Introduction}

Financial markets are inherently non-stationary. Relationships between prices, order flow, liquidity, and execution costs evolve as market structure, participant behavior, and regulatory constraints change. As a result, trading and hedging strategies calibrated under one liquidity regime are routinely deployed in environments whose statistical and economic properties differ materially from those observed during training. Episodes such as the March 2020 “dash for cash” illustrate how liquidity conditions can deteriorate abruptly, with bid-ask spreads widening as intermediation capacity becomes constrained \citep{duffie2020intermediation}.

Episodes such as the March 2020 ``dash for cash'' motivate the possibility that liquidity proxies can change economic meaning when intermediation constraints bind. Whether such sign instabilities arise empirically in a given market is ultimately an empirical question, and we do not claim a measurement here. Instead, we formalize the mechanism and test its decision-theoretic implications in a controlled identification environment.

Recent approaches, including deep hedging, learn hedging policies directly from simulated or empirical data using flexible function approximation, allowing the hedger to condition actions on rich state variables beyond classical Greeks \citep{buehler2019deep}. This flexibility can improve performance under stable conditions, but it also introduces a Lucas-critique vulnerability \citep{lucas1976}: correlations that hold in one equilibrium need not remain valid, and may invert, when the underlying trading environment changes. As learning-based hedging transitions from research prototypes toward production workflows, regime-shift vulnerabilities become a risk-management concern rather than a purely academic issue.

Microstructure proxies such as trading activity, order-flow imbalance, short-horizon momentum, and bid-ask spreads are canonical examples. In liquid regimes, elevated activity may coincide with greater depth and lower marginal impact; in stressed regimes, the same indicators may reflect forced liquidation, binding balance-sheet constraints, or adverse selection, leading to sharply higher execution costs \citep{kyle1985continuous,glosten1985bid,brunnermeier2009liquidity}. A policy that internalizes the heuristic “high volume implies cheap trading” can therefore trade most aggressively precisely when execution is most expensive after a regime shift.

We formalize this phenomenon as \emph{Semantic Inversion}: an observable-equivalence (lack of identifiability) condition in which the same regime-oblivious observation implies opposite optimal trading aggressiveness responses across regimes. We refer to the subset of such ambiguous observations as the \emph{Conflict Set} (\Cref{def:semantic_inversion}). This is more severe than ordinary miscalibration, where the sign is preserved but the magnitude changes, because sign errors induce systematically wrong-direction trading rather than merely step-size error. 

We study this mechanism in a \emph{controlled identification environment} using a minimal sufficient simulator construction. Our goal is not to claim that semantic inversion is universal, but to establish a conditional result: \emph{if} sign-instability is present, \emph{then} the information-theoretic mechanism we characterize provides a principled basis for the performance--robustness trade-off. Any empirical microstructure proxy whose economic meaning depends on endogenous liquidity conditions can, in principle, exhibit this behavior when intermediation constraints shift. By isolating this channel cleanly, we can test the proposed mechanism directly, leaving broader empirical validation to future work.

The practical dilemma is whether to incorporate microstructure signals that improve baseline execution but may invert under stress, or to accept lower in-regime efficiency in exchange for robustness. Semantic inversion makes this trade-off first-order, as wrong-direction trading under stress is typically more costly than moderate inefficiency in normal conditions.

We propose \emph{Occam’s Hedge}, which applies a representation-level constraint that prices information extraction through a KL-to-prior penalty, implemented through the variational information bottleneck. 

Exploiting microstructure signals with regime-dependent coefficients safely requires \emph{conditional} (regime-contingent) logic, i.e., inferring which regime applies, which increases representational entropy and is therefore explicitly priced by the information penalty \citep{tishby2000information,alemi2017deep,sims2003rational}. \footnote{Payoff-anchored exposures can be miscalibrated under jumps or volatility shocks, but their errors typically reflect parameter or distributional uncertainty, whereas certain microstructure proxies exhibit equilibrium-driven structural instability, including sign reversals.} This can be viewed as an operational Occam's Razor: among representations that fit the data, an information penalty favors those that encode less regime-contingent structure and are therefore less exposed to fragile equilibrium-specific correlations \citep{geirhos2020shortcut}.

Economically, the information constraint is grounded in rational inattention \citep{sims2003rational}. The penalty parameter $\beta$ represents the shadow price of maintaining high-bandwidth, regime-contingent hedging infrastructure. In our experiments $\beta$ is treated as a fixed design parameter; allowing $\beta$ to vary with market noisiness or volatility is a natural extension but orthogonal to the mechanism we isolate here. Practically, maintaining regime-contingent execution logic requires monitoring systems, data pipelines, and organizational coordination, costs that are naturally summarized by this information price. When the cost of resolving regime ambiguity exceeds the marginal benefit of microstructure-aware execution, the optimal policy leans towards a simpler, payoff-anchored strategy. Viewed this way, semantic inversion constitutes a concrete, testable failure mode that complements model-risk governance frameworks concerned with performance breakdown under changing conditions.

The remainder of the paper is organized as follows. \Cref{sec:framework} formalizes semantic inversion and the regime-information requirement, and presents Occam's Hedge as an information-constrained stochastic control policy. \Cref{sec:methodology} specifies the controlled simulator and experimental design, and \Cref{sec:experiments} reports evaluation metrics and diagnostics, including the oracle disambiguation test.

\subsection*{Contributions}
\begin{enumerate}
 \item \textbf{Conceptual.} We formalize \emph{semantic inversion} (\Cref{def:semantic_inversion}) as a representation-level failure mode distinct from magnitude miscalibration: the \emph{direction} of the optimal aggressiveness response to an observable flips across regimes. The \emph{Conflict Set} is the locus of non-identifiability where regime-oblivious policies incur irreducible ambiguity (\Cref{prop:irreducible_ambiguity}). We introduce an operational \emph{latent fragility} definition (\Cref{def:latent_fragility}) to make this claim testable.
 
 \item \textbf{Mechanism.} We establish a \emph{regime information requirement} (\Cref{prop:regime_tax}): avoiding wrong-way trading under semantic inversion requires encoding $\Omega(1)$ bits of regime information, which is explicitly priced by the VIB KL-to-prior penalty (\Cref{cor:regime_info_bound}). We implement this through \emph{hierarchical information regularization}, a structural prior that assigns different information budgets to payoff-anchored channels (low $\beta$) versus equilibrium-derived channels (high $\beta$), reflecting the different information-cost economics of these signal classes. This provides a principled basis for why information constraints suppress brittle regime-contingent logic.
 
 \item \textbf{Empirical.} In a controlled identification environment where regime information is unavailable from snapshots by design, we document: (i) near-identical baseline degradation across all representations (Deg $\approx 1.16$), confirming microstructure provides no exploitable advantage without regime labels; (ii) an information--robustness frontier where tightening $\beta$ reduces wrong-way exposure from $+0.11$ to $\approx 0$ and degradation from 1.16 to 1.0, at the cost of 60\% higher baseline risk (\Cref{tab:vib_sweep}); and (iii) an oracle disambiguation test where providing regime labels eliminates degradation entirely (Deg $= 0.97$), confirming the failure is purely informational and the model class has sufficient capacity (\Cref{tab:wrong_way}). The policy class (neural network with VIB encoder) is not the contribution; the economic logic connecting information constraints to regime-contingent fragility is.
\end{enumerate}

\begin{table}[t]
\centering
\caption{Key notation (see \Cref{app:notation} for complete list).}
\label{tab:notation_quick}
\small
\begin{tabular}{ll}
\toprule
Symbol & Meaning \\
\midrule
$R$ & Latent regime $\in\{0,1\}$, drawn per episode \\
$X_t$ & Observable state at time $t$ (excludes $R$) \\
$Z_t$ & Learned representation, $Z_t \sim q_\phi(\cdot \mid X_t)$ \\
$\beta$ & Information price (shadow cost of bandwidth) \\
$\mathrm{ES}_{0.95}$ & Expected Shortfall at 95\% level \\
$R_0, R_1$ & Regime-conditional $\mathrm{ES}_{0.95}$ \\
$\mathrm{Deg}$ & Degradation ratio $R_1/R_0$ \\
\bottomrule
\end{tabular}
\end{table}

\section{Literature Review}
\label{sec:literature}
Our framework sits at the intersection of four strands: (i) deep hedging under frictions, (ii) microstructure-driven execution costs and regime dependence, (iii) model risk and robustness in finance, and (iv) information-constrained representations and shortcut learning under distribution shift. The unifying theme is \emph{semantic stability}: features whose economic meaning is equilibrium-dependent can be predictive in-sample yet brittle under regime shifts, and we study how information constraints suppress such brittleness at the representation level.

\paragraph{Deep hedging under frictions.}
Deep hedging formulates hedging with frictions as a neural stochastic control problem \citep{buehler2019deep}. Extensions address rough volatility, alternative risk measures, and RL-based formulations \citep{gonon2021deephedging,horvath2021deephedging,buehler2022deepbellman}. Empirical implementations highlight sensitivity to distributional assumptions \citep{mikkila2023empirical}. We take an orthogonal perspective: isolating a regime-shift failure mode tied to \emph{representation choice}. This vulnerability arises because ERM rewards any predictive dependence, regardless of semantic invariance \citep{geirhos2020shortcut,arjovsky2019invariant}.

\paragraph{Execution costs, microstructure, and regime dependence.}
Market impact and transaction costs arise endogenously from liquidity provision and adverse selection \citep{kyle1985continuous,glosten1985bid}. Empirically, impact is state-dependent and persistent, reflecting slowly evolving liquidity and order-flow dynamics \citep{bouchaud2008digest,gatheral2012impact}. Classical execution-cost models provide workhorse baselines for mapping trading intensity into temporary and permanent impact and for formalizing liquidity as an equilibrium object rather than a fixed ``fee''
\citep[e.g.,][]{almgren2001optimal,obizhaeva2013optimal}. Constraint-based theories link funding conditions to market liquidity, implying nonlinear deterioration when balance-sheet or margin constraints bind \citep{brunnermeier2009liquidity}. Episodes such as the March~2020 pandemic dislocation illustrate that intermediation-capacity constraints can sharply alter execution conditions even in otherwise highly liquid markets \citep{duffie2020intermediation}. When such constraints bind, order flow increasingly reflects forced liquidation rather than voluntary liquidity provision, so variables that proxy trading ``activity'' or ``flow'' cease to measure market depth and instead signal scarcity, potentially reversing their marginal effect on execution costs. These mechanisms motivate our distinction between \emph{payoff-anchored} structural signals (payoff sensitivities) and \emph{equilibrium proxies} (microstructure): the latter may be highly predictive within a regime yet change economic meaning precisely when constraints become binding. Operationally, we relate this distinction to directional stability: a signal is payoff-anchored if the benchmark-relative optimal adjustment direction (equivalently, the sign of the action-gradient of the loss) is stable across regimes on the regime-oblivious observation space, whereas an equilibrium proxy admits regime-dependent sign changes.

\paragraph{Model risk and robustness in finance.}
Robust control and model uncertainty formalize decision-making under misspecification and ambiguity \citep{hansen2008robust,cont2006model,glasserman2014modelrisk}. In derivatives and hedging, robust pricing and hedging traditions study worst-case valuation and control under model classes or ambiguity sets \citep[e.g.,][]{avellaneda1995pricing,soner2011martingale}, while recent computational work increasingly explores robustness-enhancing training procedures and stress testing of learned policies, including adversarial or distributional perturbations \citep{limmer2024robustgans}. Our contribution differs in target: rather than optimizing worst-case scenarios directly, we constrain \emph{representation bandwidth} so that the policy cannot cheaply implement brittle regime-contingent logic. This choice reflects our focus on semantic instability rather than tail risk: worst-case optimization hardens policies against extreme realizations under fixed semantics, whereas information constraints directly penalize reliance on latent regime inference, which is the source of fragility in sign-flipping microstructure features.

\paragraph{Information constraints and shortcut learning under shift.}
Information-processing limits provide a principled way to discipline complex conditional strategies by pricing state contingency \citep{sims2003rational}. In machine learning, the information bottleneck and its variational implementations formalize a trade-off between predictive accuracy and compression through a KL-to-prior penalty \citep{tishby2000information,alemi2017deep}. Closely related Occam-style perspectives link generalization to description length or KL-to-prior control \citep{grunwald2007minimum}. Separately, work on OOD generalization and shortcut learning shows that high-capacity models often exploit predictive but non-invariant features that fail under distribution shift \citep{geirhos2020shortcut,arjovsky2019invariant,sagawa2020distributionally}. In our setting, semantically unstable microstructure variables play the role of such shortcuts: using them safely requires inferring an unobserved regime. Information regularization suppresses this brittle dependence by making regime-contingent logic costly to encode. Whereas invariant risk minimization seeks features whose predictive relationship is invariant across environments \citep{arjovsky2019invariant}, Occam's Hedge addresses the case in which invariance is \emph{not} achievable on the regime-oblivious observation space by pricing the information bandwidth required to resolve non-invariant (regime-contingent) signals.

\paragraph{Distinguishing information constraints from standard regularization.}
Weight-decay ($L_2$) regularization shrinks parameters toward zero, reducing model capacity uniformly across all input dimensions. Distributionally robust optimization (DRO) hardens policies against worst-case perturbations within an ambiguity set. Occam's Hedge differs in mechanism: the VIB KL penalty operates on the \emph{representation} rather than parameters, and penalizes the \emph{information content} extracted from inputs rather than parameter magnitude. This distinction matters under semantic inversion: $L_2$ shrinks all dependencies equally, while VIB selectively suppresses high-entropy regime-contingent logic because such logic requires encoding regime-distinguishing bits to be implemented. Empirically, we compare VIB to $L_2$ baselines (\Cref{subsec:exp_reg}) and find that VIB achieves lower degradation at matched baseline risk, consistent with targeted information suppression rather than generic capacity reduction. Unlike parameter shrinkage, the VIB penalty prices \emph{state contingency}: implementing a regime-dependent sign flip on the conflict set requires the representation to encode regime-distinguishing bits. $L_2$ regularization reduces sensitivity uniformly, but does not preferentially penalize the acquisition of regime information; hence it need not suppress wrong-way trading at matched in-regime performance.

\section{Model and Theoretical Framework}
\label{sec:framework}

\begin{quote}
\textbf{Conditional claim (scope).} We do not claim semantic inversion is universal. Our claim is: \emph{if} execution-cost proxies admit regime-dependent sign changes on a regime-oblivious observation space, \emph{then} avoiding wrong-way trading requires regime information, and pricing that information yields a principled efficiency--robustness trade-off.
\end{quote}

This section reframes regime-shift fragility as an information-economic problem. The core mechanism is not merely statistical distribution shift, but a Lucas-critique breakdown in microstructure: when liquidity is an endogenous equilibrium outcome, the mapping from observables (e.g.\ trading activity) to execution costs can change sign when intermediation constraints bind. Here the ``policy change'' is a shift in intermediation constraints that alters the equilibrium mapping from observables to execution costs; the object that breaks is not a physical law but an endogenous microstructure relationship. We formalize this as \emph{Semantic Inversion} and show that avoiding wrong-way trading imposes a \emph{regime information requirement}. We call the subset of observations on which the \emph{optimal trading aggressiveness response} differs across regimes the \emph{Conflict Set}; it is the locus of non-identifiability driving wrong-way behavior (formalized in \Cref{def:semantic_inversion}). Occam's Hedge is then presented as an information-constrained stochastic control policy: the VIB KL-to-prior penalty is a tractable numerical surrogate for rational inattention. Throughout, Semantic Inversion is treated as a decision-theoretic non-identifiability problem: conditional on the regime-oblivious observation space, the optimal adjustment is not a single-valued function of the input.


\subsection{Problem Setup}
\label{subsec:setup}

We consider discrete-time hedging at times $t = 0, 1, \ldots, T$ and denote $\mathcal{T}$ as index set of trading times. Let $S_t$ denote the underlying price. The hedger holds a position $a_t\in\mathbb{R}$, with trades $\Delta a_t := a_t - a_{t-1}$ and $a_{-1}:=0$. The terminal liability is $L$. Market conditions are indexed by a latent regime $R\in\{0,1\}$ drawn once per episode and held fixed throughout. The policy does not observe $R$. At each time $t$ it observes a regime-oblivious state $X_t=\psi(\tilde S_t)$, a coarsening of the full market state $\tilde S_t$ that intentionally excludes the latent regime label and its sufficient identifiers. This prevents the policy from trivially recovering $R$ from $X_t$ and forces the representation $Z_t$ to act as the sole pipeline for any regime-contingent information. The agent acts through a (possibly stochastic) representation:

\[
Z_t \sim q_\phi(\cdot \mid X_t),\qquad a_t = \pi_\theta(Z_t),
\]
where $\phi$ parameterizes the encoder and $\theta$ the trading rule.

\vspace{0.3cm}

\noindent\emph{Remark.} We draw $R$ once per episode to isolate regime-dependent semantics in a controlled setting. Allowing within-episode switching introduces additional partial observability and is deferred to future work.

\subsection{Execution Costs and Semantic Inversion}
\label{subsec:semantic_inversion}

Trading incurs regime- and state-dependent execution costs $C_{r,t}(\Delta a_t; X_t)$. We use a spread-plus-quadratic-impact model:
\begin{equation}
C_{r,t}(\Delta a_t; X_t)
= c_{\mathrm{spread}}|\Delta a_t|
+ \frac{1}{2}\lambda_{r,t}(X_t)(\Delta a_t)^2 .
\label{eq:cost}
\end{equation}
Here $\lambda_{r,t}(X_t)$ summarizes the effective marginal impact faced by the hedger. Regime dependence enters through $\lambda_{r,t}(X_t)$ and can induce regime-contingent optimal trading adjustments relative to a frictionless benchmark, motivating the notion of semantic inversion.

\begin{definition}[Semantic inversion and the conflict set]
\label{def:semantic_inversion}

Let $\mathcal X$ denote the space of regime-oblivious, decision-time observables $x$ available to the agent. Typical components of $x$ include quantities such as $(\tau_t, \log(S_t/K), a_{t-1}, \mathrm{Vol}_t, \ldots)$. For each regime $r \in \{0,1\}$, let $\pi_r^*(x)$ denote the regime-optimal action given $x$. Define the associated \emph{optimal trading aggressiveness} by
\[
u_r^*(x) := \big|\pi_r^*(x) - a_{t-1}\big|,
\]
where $a_{t-1}$ is included as a component of $x$. We restrict attention to points $x$ where $u_r^*(x)$ is differentiable in $x_j$ (e.g., away from the
no-trade kink where $\pi_r^*(x)=a_{t-1}$).

Fix a feature coordinate $x_j$ and a threshold $\delta > 0$. We say that $x_j$ exhibits \emph{Semantic Inversion} between regimes $r=0$ and $r=1$ if there exists a measurable \emph{Conflict Set} $\mathcal X_{\mathrm{conflict}} \subset \mathcal X$ such that
\[
\mathbb P_0(\mathcal X_{\mathrm{conflict}}) > 0
\quad \text{and} \quad
\mathbb P_1(\mathcal X_{\mathrm{conflict}}) > 0,
\]
and for all $x \in \mathcal X_{\mathrm{conflict}}$,
\begin{equation}
\frac{\partial u_0^*(x)}{\partial x_j}
\cdot
\frac{\partial u_1^*(x)}{\partial x_j}
< 0,
\qquad
\min\!\left\{
\left|\frac{\partial u_0^*(x)}{\partial x_j}\right|,
\left|\frac{\partial u_1^*(x)}{\partial x_j}\right|
\right\}
\ge \delta.
\label{eq:semantic_inversion_aggr}
\end{equation}

\end{definition}

\noindent\emph{Remark.} We distinguish semantic inversion (a sign flip) from magnitude instability (a scale change with preserved sign), focusing on the former because it induces directionally wrong trading rather than merely miscalibrated step sizes.



\begin{lemma}[Damped adjustment under quadratic impact]
\label{lem:damped_adjustment}

Consider a one-step quadratic approximation in which the agent trades from $a_{t-1}$ toward a frictionless target $a_f(x)$ by choosing an action $a$.

Suppose the local objective takes the form
\[
\min_{a\in\mathbb R}\ (a-a_f(x))^2 \;+\; \lambda(x)\,(a-a_{t-1})^2,
\qquad \lambda(x)>0.
\]

Then the optimizer is given by the damped adjustment
\[
a^*(x)=\frac{a_f(x)+\lambda(x)a_{t-1}}{1+\lambda(x)},
\qquad
u^*(x):=|a^*(x)-a_{t-1}|=\frac{|a_f(x)-a_{t-1}|}{1+\lambda(x)}.
\]

In particular, fix a feature coordinate $x_j$ such that $a_f(x)$ does not depend on $x_j$ locally (i.e., $\partial a_f(x)/\partial x_j=0$ in a neighborhood) and such that $|a_f(x)-a_{t-1}|>0$ locally. Then
\[
\mathrm{sgn}\!\left(\frac{\partial u^*(x)}{\partial x_j}\right)
=
-\mathrm{sgn}\!\left(\frac{\partial \lambda(x)}{\partial x_j}\right),
\]
whenever the derivatives exist.

\end{lemma}

\paragraph{Implication for controlled inversion construction}

In our simulator, the impact elasticity satisfies
\[
\mathrm{sgn}\!\left(\frac{\partial \lambda_{0,t}}{\partial \mathrm{Vol}_t}\right)
\neq
\mathrm{sgn}\!\left(\frac{\partial \lambda_{1,t}}{\partial \mathrm{Vol}_t}\right)
\]
by \cref{eq:lambda_inversion}.

By \Cref{lem:damped_adjustment}, the optimal aggressiveness response to $\mathrm{Vol}_t$ therefore flips sign across regimes on any overlap set where the frictionless adjustment magnitude $|a_f(x)-a_{t-1}|$ is not degenerate.

This behavior is exactly the directional-instability content of \Cref{def:semantic_inversion}.

\begin{proposition}[Irreducible ambiguity under regime-oblivious observations]
\label{prop:irreducible_ambiguity}
Let
\[
\mathcal R_r(\pi)
\;:=\;
\mathbb E_{x \sim \mathbb P_r}\!\left[\,\ell_r\!\big(x,\pi(x)\big)\right]
\]
denote the regime-$r$ expected hedging risk under policy $\pi$, where $\mathbb P_r$ denotes the distribution of the observable $x$ under regime $r$. If $\mathcal{X}_{\mathrm{conflict}}$ has positive probability under both regimes, then there exists no measurable regime-oblivious policy $\pi:\mathcal{X}\to\mathbb{R}$ that is optimal for both regimes simultaneously. In particular, for any regime-oblivious $\pi$ there exists $r\in\{0,1\}$ such that
\[
\mathcal{R}_r(\pi)-\mathcal{R}_r(\pi^*_r)>0.
\]
\end{proposition}

\paragraph{Interpretation}
On $\mathcal X_{\mathrm{conflict}}$, the direction of the optimal aggressiveness response to the same observed feature differs across regimes: the policy should trade more when $x_j$ increases in one regime and trade less when $x_j$ increases in the other. Therefore, any regime-oblivious policy that conditions on $x$ alone cannot be simultaneously optimal across regimes on the overlap region. This constitutes a Lucas-critique vulnerability, since the observable feature does not retain a stable economic meaning for execution costs across equilibria and the optimal trading response is not invariant under structural change.

\vspace{0.3cm}
\noindent\emph{Proof.} See \Cref{app:proof-prop_ambiguity}.


\vspace{0.3cm}
\noindent\emph{Remark on magnitude instability.}
We distinguish semantic inversion (sign flip) from magnitude instability
(regime-dependent coefficient strength with preserved sign). While both affect
robustness, only semantic inversion induces directionally wrong trading.
Entropy regularization selectively suppresses semantically inverted features
while remaining tolerant to magnitude changes.


\subsection{Regime Information}
\label{subsec:regime_tax}

The key mechanism is \emph{informational}. Under inversion, using a feature safely requires \emph{regime-contingent} action logic, which in turn requires encoding information about the latent regime. Thus wrong-way trading is not primarily a modeling failure; it is an \emph{information failure}. We isolate the logic in the simplest one-step setting.

To isolate the informational requirement implied by sign-flipping marginal costs, we study a one-step proxy problem in which the optimal action direction depends on an unobserved regime-dependent sign.

\begin{proposition}[Regime information requirement]
\label{prop:regime_tax}
Consider a one-step problem with latent regime $R\in\{0,1\}$, $\mathbb{P}(R=0)=\mathbb{P}(R=1)=\frac12$, and an observed directional signal $V\in\{-1,+1\}$ (e.g., the sign of a frictionless hedge adjustment), whose economic interpretation depends on $R$. Define the regime-dependent sign

$s_R$ by $s_0=+1$ and $s_1=-1$, and consider the squared loss
\[
\ell(a;R,V) := (a - s_R V)^2.
\]
Let $Z$ be a representation and $a=\pi(Z)$ be a policy with $\pi(Z)\in\{-1,+1\}$ almost surely. Define the implied sign choice $\hat s(Z):=\mathrm{sgn}(\pi(Z))\in\{-1,+1\}$ (with $\mathrm{sgn}(0)=+1$). If $\mathbb{E}[\ell(\pi(Z);R,V)] \le \epsilon$, then the wrong-way probability
\[
p_e := \mathbb{P}\big(\hat s(Z)\neq s_R\big)
\]
satisfies $p_e \le \epsilon/4$. Moreover, the mutual information obeys
\[
I(R;Z) \ge \log 2 - h(p_e),
\]
where $h(\cdot)$ is the binary Shannon entropy function (with $\log 2$ equal to one bit under base-2
logs, or $\ln 2$ nats under natural logs).
\end{proposition}

\begin{proof}[Proof sketch]
Let $\hat s(Z):=\mathrm{sgn}(\pi(Z))\in\{-1,+1\}$ denote the implied sign choice. Since $V\in\{\pm 1\}$ is observed, the only ambiguity is the regime-dependent sign $s_R$; thus $\{\hat s(Z)\neq s_R\}$ corresponds to choosing the wrong sign for the target $s_RV$. Because $\pi(Z)\in\{\pm 1\}$ and $s_RV\in\{\pm 1\}$, we have
\[
(\pi(Z)-s_RV)^2 =
\begin{cases}
0, & \pi(Z)=s_RV,\\
4, & \pi(Z)=-s_RV.
\end{cases}
\]
Hence $\mathbb{E}[\ell(\pi(Z);R,V)] = 4\,p_e$, so $p_e \le \epsilon/4$. For the information bound, define a decoder $\hat R(Z)$ via the bijection between $R$ and $s_R$ (i.e., $\hat R(Z)=0$ if $\hat s(Z)=+1$ and $\hat R(Z)=1$ if $\hat s(Z)=-1$). Then $\mathbb{P}(\hat R(Z)\neq R)=p_e$, and by the binary case of Fano's inequality \citep{cover2006elements}, $H(R\mid Z)\le h(p_e)$. Since $H(R)=\log 2$, we obtain $I(R;Z)=H(R)-H(R\mid Z)\ge \log 2 - h(p_e)$.
\end{proof}

\paragraph{Interpretation.}
When $\epsilon$ is small, $p_e$ is small and the bound implies $I(R;Z)\approx \log 2$: nearly one bit of regime information is required to avoid wrong-way behavior under inversion. Equivalently, any compressed representation $Z$ with $I(R;Z)\ll 1$ bit cannot implement regime-contingent sign logic, and will exhibit wrong-way behavior under inversion. This is a \emph{fundamental information requirement} for safe trading in the presence of regime-dependent semantics.

\vspace{0.3cm}
\noindent\emph{Mechanism closure.}
\Cref{prop:regime_tax} provides a lower bound on the regime information needed to avoid wrong-way behavior. Combined with the KL-to-prior control on $I(X;Z)$ (via \Cref{prop:kl_upper_bound} in the next subsection) and data processing $I(R;Z)\le I(X;Z)$, it implies that sufficiently strong compression necessarily prevents the policy from reliably encoding regime-contingent sign logic on conflict states. Hence, as the information price increases, the policy must rationally shed dependence on semantically inverted drivers first.

\begin{remark}[Application to information-starved designs]
\Cref{prop:regime_tax} establishes a fundamental information requirement for avoiding wrong-way behavior under semantic inversion. In experimental designs where regime information is unavailable from observations ($I(R;X_t) \approx 0$), such as our variance-matched snapshot setting (\Cref{subsec:variance_matched}), the bound implies wrong-way behavior is \emph{unavoidable} without supplying additional information sources (e.g., temporal integration over regime-revealing dynamics, or oracle regime labels). In such settings, the role of information constraints is not to reduce available regime information (which is already minimal), but to prevent policies from learning regime-contingent logic that cannot be reliably implemented given the available information.
\end{remark}

\begin{definition}[Wrong-way trading score (aggressiveness vs.\ marginal costs)]
\label{def:wrong_way}
Fix a microstructure driver $x_j$ (e.g.\ $\mathrm{Vol}_t$) that affects marginal impact $\lambda_{r,t}$. Define aggressiveness $u_t:=|\Delta a_t|$. Let $H_q(r)$ and $L_q(r)$ denote the sets of times where $x_{j,t}$ lies in the top/bottom $q$-quantile \emph{within regime $r$} (quantiles computed regime-conditionally). We denote the wrong-way trading score by $W_r := \mathrm{WWScore}_r$ for notational convenience. Define
\[
W_r
:=
\mathrm{sgn}\!\left(\frac{\partial \lambda_{r}(x)}{\partial x_j}\right)\cdot
\left(\mathbb{E}[u_t\mid R=r,\ t\in H_q(r)]-\mathbb{E}[u_t\mid R=r,\ t\in L_q(r)]\right),
\]
where $\partial \lambda_r/\partial x_j$ is evaluated under the regime-$r$ cost specification and is assumed to have a well-defined nonzero sign under that specification.
\end{definition}

\paragraph{Interpretation.}
If $\partial \lambda_r/\partial x_j>0$, higher values of $x_j$ imply higher marginal execution costs, so right-way behavior reduces aggressiveness at high $x_j$. If $\partial \lambda_r/\partial x_j<0$, higher values of $x_j$ imply lower marginal execution costs, so right-way behavior increases aggressiveness at high $x_j$. By construction, $\mathrm{WWScore}_r>0$ indicates wrong-way behavior in either case (trading more when marginal costs are higher), while $\mathrm{WWScore}_r<0$ indicates right-way behavior.

\begin{definition}[Latent Fragility under Semantic Inversion]
\label{def:latent_fragility}
A feature set $\mathcal{F}$ exhibits \emph{latent fragility} relative to baseline features $\mathcal{F}_0$ if:
\begin{enumerate}
 \item[(i)] \textbf{Baseline indifference:} In-regime performance is indistinguishable, $R_0(\mathcal{F}) \approx R_0(\mathcal{F}_0)$;
 \item[(ii)] \textbf{Stress fragility:} Out-of-regime performance degrades materially, $R_1(\mathcal{F}) > R_1(\mathcal{F}_0) + \delta$ for $\delta > 0$;
 \item[(iii)] \textbf{Mechanistic link:} The degradation is attributable to $\mathcal{F}$ containing signals that exhibit semantic inversion (\Cref{def:semantic_inversion}), manifesting as positive wrong-way trading scores $W_1 > 0$ (\Cref{def:wrong_way}).
\end{enumerate}
Under this definition, a practitioner selecting models by in-regime validation cannot distinguish $\mathcal{F}$ from $\mathcal{F}_0$, this is a governance and validation failure. The fragility is \emph{latent}: invisible under baseline conditions but manifesting under regime shift.
\end{definition}

\paragraph{Why latent fragility matters under variance-matching.}
Under our variance-matched identification design (\Cref{subsec:variance_matched}), all representations achieve approximately equal baseline performance $R_0$. The fragility is therefore not visible in standard in-regime validation. It manifests only under stress, when the semantic relationship between microstructure and execution costs inverts. This creates a model-selection failure mode: practitioners cannot identify the vulnerable model ex ante using regime-oblivious validation procedures.

%==========================================================================

\subsection{Hedging Error and Risk}
\label{subsec:risk}

Given $(\pi_\theta,q_\phi)$, define the terminal hedging error under regime $r$:
\begin{equation}
Y^{\theta,\phi}_r
:= L - \sum_{t=0}^{T-1} a_t(S_{t+1}-S_t) - \sum_{t=0}^{T-1} C_{r,t}(\Delta a_t;X_t).
\label{eq:terminal_error}
\end{equation}
We evaluate tail risk using Conditional Value-at-Risk (CVaR) at level $\gamma$ (also referred to as Expected Shortfall under standard regularity). We use the Rockafellar--Uryasev representation \citep{rockafellar2000optimization,rockafellar2002conditional}:

\begin{equation}
\mathrm{CVaR}_\gamma(Y)
=
\min_{\eta\in\mathbb R}\left\{
\eta + \frac{1}{1-\gamma}\,\mathbb E\big[(Y-\eta)_+\big]
\right\}.
\label{eq:cvar_ru}
\end{equation}
Accordingly, we define the regime-$r$ risk functional as
\[
\mathcal{R}_r(\theta,\phi) := \mathrm{CVaR}_\gamma\!\left(Y^{\theta,\phi}_r\right).
\]
In experiments we report $\mathrm{ES}_{0.95}$ computed empirically on held-out paths; for continuous losses, this coincides with $\mathrm{CVaR}_{0.95}$, and \cref{eq:cvar_ru} provides stable gradients during training.


\noindent\emph{Remark.} While Expected Shortfall is convex in $Y$ and the KL term introduced below is convex in $q_\phi(\cdot\mid X)$, the overall training problem is nonconvex in neural parameters $(\theta,\phi)$.

\paragraph{Risk Sensitivity to Directional Errors.}
Unlike magnitude miscalibration (scale errors), semantic inversion induces \emph{directional} mistakes:
the policy trades in the wrong adjustment direction on the Conflict Set. Such errors can generate large
execution costs and concentrated losses, disproportionately affecting the tail of the hedging error
distribution. We therefore evaluate policies using Expected Shortfall ($\text{ES}_{0.95}$), which is designed
to capture catastrophic tail breakdown that mean-squared objectives can mask.

\paragraph{Equilibrium-dependent semantics and informational bias toward simplicity.}
Semantic inversion represents an equilibrium shift: the observable $x$ no longer summarizes the same
structural object (e.g., liquidity) across regimes. Implementing a regime-switching adjustment flip on
$\mathcal{X}_{conflict}$ therefore requires encoding regime-disambiguating information in $Z$.
The VIB KL-to-prior penalty (\Cref{eq:occams_objective}) explicitly prices these regime-specific bits.
As $\beta$ increases, the optimal encoder rationally sheds brittle, high-entropy regime-contingent
dependencies first, biasing the learned strategy toward smoother, regime-invariant payoff features
(e.g., Greeks), even if those remain imperfect in magnitude due to misspecified inputs and altered distribution under stress.

%==========================================================================

\subsection{Occam's Hedge as Information-Constrained Stochastic Control}
\label{subsec:occams_hedge}

Occam's Hedge treats the hedger as a \emph{rationally inattentive} controller: the agent faces a shadow price for representational bandwidth, so regime-contingent logic is costly to maintain. This is the economic complement to \Cref{prop:regime_tax}: if safe use of inverted signals requires regime information, then an information-constrained agent should rationally default toward structural invariants (payoff-anchored features) when regime ambiguity is too expensive to resolve.

We implement this information constraint via a variational information bottleneck encoder:
\[
q_\phi(z_t\mid X_t)=\mathcal{N}\!\big(\mu_\phi(X_t),\mathrm{diag}(\sigma_\phi^2(X_t))\big),\qquad
p(z)=\mathcal{N}(0,I),
\]
where $p(z)$ denotes a fixed reference distribution corresponding to a minimally informative (no-attention) representation, and $I$ denotes the identity matrix, corresponding to independent unit-variance noise across latent components. 

Define the time-averaged information cost:
\begin{equation}
\mathcal{C}(\phi)
:= \mathbb{E}\left[\frac{1}{T}\sum_{t=0}^{T-1}
\mathrm{KL}\!\left(q_\phi(Z_t\mid X_t)\,\|\,p(Z_t)\right)\right].
\label{eq:info_cost}
\end{equation}
The expectation is taken over the mixed-regime training distribution (50\% Regime~0, 50\% Regime~1 episodes). For interpretability, we also report the regime-conditional realized cost $\mathcal{C}_r(\phi)$ when evaluating performance separately on Regime~0 and Regime~1 test sets.

\paragraph{Snapshot versus history.}
In our implementation, hedging decisions are functions of $Z_t$ (and position variables) without an additional unconstrained memory channel. Consistent with Assumptions~\ref{ass:hierarchy}--\ref{ass:exogenous_costs}, the information bottleneck is applied to an instantaneous representation $Z_t \sim q_\phi(\cdot \mid X_t)$, so that regime-contingent distinctions requiring temporal integration cannot be encoded without incurring sustained information cost.

\vspace{0.3cm}

\noindent We implement a two-channel encoder reflecting the structural prior (\Cref{subsec:variance_matched}):
\[
Z^{\mathrm{price}}_t \sim q_{\phi_p}(\cdot\mid X^{\mathrm{price}}_t),\qquad
Z^{\mathrm{micro}}_t \sim q_{\phi_m}(\cdot\mid X^{\mathrm{micro}}_t),
\]
with priors $p(z)=\mathcal N(0,I)$ for each channel. Define the per-channel information costs:
\[
\mathcal C_{\mathrm{price}}(\phi_p)
:= \mathbb{E}\!\left[\frac1T\sum_{t=0}^{T-1}\mathrm{KL}\!\left(q_{\phi_p}(Z^{\mathrm{price}}_t\mid X^{\mathrm{price}}_t)\,\|\,p(Z)\right)\right],
\quad
\mathcal C_{\mathrm{micro}}(\phi_m)
:= \mathbb{E}\!\left[\frac1T\sum_{t=0}^{T-1}\mathrm{KL}\!\left(q_{\phi_m}(Z^{\mathrm{micro}}_t\mid X^{\mathrm{micro}}_t)\,\|\,p(Z)\right)\right].
\]
The training objective is the mixture-CVaR with hierarchical information penalty:
\begin{equation}
\min_{\theta,\phi_p,\phi_m,\eta}\ \ 
\left\{\eta + \frac{1}{1-\gamma}\,\mathbb E\big[(Y^{\theta,\phi_p,\phi_m}_{\mathrm{mix}}-\eta)_+\big]\right\}
\;+\; \beta_{\mathrm{price}}\,\mathcal{C}_{\mathrm{price}}(\phi_p)
\;+\; \beta_{\mathrm{micro}}\,\mathcal{C}_{\mathrm{micro}}(\phi_m),
\label{eq:occams_objective}
\end{equation}
where $Y^{\theta,\phi_p,\phi_m}_{\mathrm{mix}}$ is the terminal hedging error under the mixture distribution, and the shadow prices $\beta_{\mathrm{price}} \ll \beta_{\mathrm{micro}}$ reflect the structural prior: payoff-anchored channels (Greeks) are essentially uncompressed, while equilibrium-derived channels (microstructure) are priced according to their regime-disambiguation cost.

\begin{proposition}[KL-to-prior upper-bounds representation information]
\label{prop:kl_upper_bound}
Let $q_\phi(z\mid x)$ be any encoder and let $q_\phi(z):=\mathbb{E}_{R,X|R}[q_\phi(z\mid X)]$ denote the aggregated posterior under the mixed-regime training distribution. For any prior $p(z)$,
\[
\mathbb{E}\Big[\mathrm{KL}(q_\phi(Z\mid X)\,\|\,p(Z))\Big]
= I(X;Z) + \mathrm{KL}(q_\phi(Z)\,\|\,p(Z)).
\]
In particular, $\mathbb{E}[\mathrm{KL}(q_\phi(Z\mid X)\,\|\,p(Z))]\ge I(X;Z)$.
\end{proposition}

\paragraph{Interpretation.}
By \Cref{prop:kl_upper_bound}, the KL-to-prior penalty equals $I(X;Z)$ plus the nonnegative aggregated-posterior mismatch term $\mathrm{KL}(q_\phi(Z)\,\|\,p(Z))$, and hence upper-bounds $I(X;Z)$. Thus minimizing the KL term limits representational bandwidth, and $\beta$ acts as a Lagrange multiplier pricing the agent's effective information-processing rate.

\begin{corollary}[Regime information is also bounded]
\label{cor:regime_info_bound}
If the latent regime $R$ influences observations $X$ and the representation is generated as $R\to X\to Z$ (i.e.\ $Z\sim q_\phi(\cdot\mid X)$), then by data processing,
\[
I(R;Z)\le I(X;Z)\le \mathbb{E}\big[\mathrm{KL}(q_\phi(Z\mid X)\,\|\,p(Z))\big].
\]
Thus increasing $\beta$ constrains the capacity of $Z$ to carry regime information when such information is present in the inputs.
\end{corollary}

\noindent\emph{Interpretation.} In settings where $I(R;X_t) > 0$, increasing $\beta$ reduces the capacity of $Z$ to carry regime information via data processing. In our variance-matched snapshot design where $I(R;X_t) \approx 0$ by construction (\Cref{subsec:variance_matched}), the implication is different: increasing $\beta$ prevents policies from learning representations that would \emph{require} regime information to use safely, when that information is fundamentally unavailable from the observation space.

\paragraph{Mechanism summary.}\label{subsec:mechanism}
\Cref{prop:regime_tax} establishes that avoiding wrong-way behavior under semantic inversion requires encoding regime-disambiguating information in the representation. \Cref{cor:regime_info_bound} shows that the VIB KL penalty upper-bounds $I(X;Z)$ and, by data processing, also bounds $I(R;Z)$. When regime information is unavailable from observations ($I(R;X_t) \approx 0$), \Cref{prop:regime_tax} implies safe exploitation of inverted signals is impossible without additional information. As $\beta$ increases, the optimal encoder avoids learning dependencies on semantically unstable signals that presume access to regime information the observation space cannot provide, rationally biasing toward payoff-anchored features whose decision-theoretic optimality does not depend on which equilibrium obtains. We do not assume a perfectly regime-invariant optimum; the claim is that compression makes regime-contingent logic increasingly uneconomical when the required regime information is unavailable.


%====================================================================================

\subsection{Testable Implications}
\label{subsec:implications}

The information-economic mechanism yields testable predictions: (i) microstructure inputs with regime-dependent semantics exhibit semantic inversion (\Cref{def:semantic_inversion}); (ii) as $\beta$ increases, dependence on semantically unstable signals decreases (lower wrong-way exposure); (iii) supplying an oracle regime label restores the value of microstructure inputs by providing the required regime information exogenously (``oracle disambiguation'').

\paragraph{Training regime.}
Training follows the mixed-regime protocol (\Cref{tab:protocol}); robustness differences are interpreted as Lucas-critique mitigation via information constraints rather than improved regime classification.

%============================================================================


\subsection{Modeling assumptions}
\label{subsec:assumptions}

\paragraph{Modeling assumptions.}
Our simulator embodies three premises: (i) some signals are payoff-anchored (Greeks) while others are equilibrium proxies (microstructure), the latter requiring regime inference for safe use; (ii) regimes persist within episodes, preventing trivial identification from single snapshots; and (iii) individual agent trading does not endogenously manipulate regime dynamics. Formal statements are provided in \Cref{app:assumptions}.

%============================================================================

\section{Methodology}
\label{sec:methodology}
We design a minimal simulator that produces latent fragility: microstructure features achieve comparable in-sample risk in a baseline regime but become harmful after a \emph{semantic inversion} in execution-cost elasticities. The goal is to prioritize clarity and implementability over maximal market realism.

\paragraph{Regimes and evaluation protocol.} Each episode is assigned a regime $r\in\{0,1\}$ drawn uniformly at random; the regime is constant within an episode but \emph{not observed} by the agent. During training, the agent observes episodes from both regimes (50% each) without access to the regime label $R$. This \emph{mixed-regime training} forces the agent to develop policies that perform across both regimes; failure to encode regime-contingent logic manifests as wrong-way trading on conflict states. Evaluation is performed on held-out episodes from each regime separately, enabling regime-conditional performance measurement. Throughout, $\{\epsilon^S_t\}_t$ and $\{\epsilon^v_t\}_t$ are i.i.d.\ standard Gaussians and independent of each other.

\noindent\emph{Notation:} We use $R$ to denote the latent regime random variable; $r \in \{0,1\}$ to index regime realizations; $R_0$ and $R_1$ to denote regime-conditional ES evaluation metrics; and ``mix'' subscripts to denote the training distribution over episodes with $R \sim \mathrm{Bernoulli}(0.5)$.


\subsection{Underlying Dynamics}
\label{subsec:heston}
The underlying price $S_t$ and its instantaneous variance $v_t$ follow a Heston stochastic volatility model discretized via a full-truncation Euler scheme:
\begin{equation}
\label{eq:heston_s}
S_{t+1} = S_t \exp\left( (\mu - \tfrac{1}{2}\bar{v}_t)\Delta t + \sqrt{\bar{v}_t \Delta t} \epsilon^S_t \right)
\end{equation}
\begin{equation}
\label{eq:heston_v}
v_{t+1} = \bar{v}_t + \kappa(\theta - \bar{v}_t)\Delta t + \xi \sqrt{\bar{v}_t \Delta t} \epsilon^v_t
\end{equation}
where $\bar{v}_t = \max(v_t, \text{eps})$ and the shocks are correlated with $d\langle W^S, W^v \rangle_t = \rho dt$.
We use baseline parameters $\kappa=2.0$, $\theta=0.04$ (implying 20\% long-run volatility), $\xi=0.60$, and $\rho=-0.60$ to reflect realistic equity-like dynamics. The liability is a European payoff $L=\max(S_T-K,0)$, and Black--Scholes Greeks are computed using the long-run volatility $\sqrt{\theta}$.

\subsection{Observed Activity Proxy}
\label{subsec:vol_proxy}
We model the activity proxy as an exogenous lognormal process that is \emph{explicitly independent} of the price and variance dynamics, ensuring clean identification of the semantic inversion channel. The observed proxy is defined as:
\begin{equation}
\label{eq:vol_model}
\mathrm{Vol}_t = \exp\left( \mu_v + \sigma_{v,r} \epsilon^{\mathrm{vol}}_t \right), \qquad \epsilon^{\mathrm{vol}}_t \sim \mathcal{N}(0, 1),
\end{equation}
where $\epsilon^{\mathrm{vol}}_t$ is independent of the Brownian motions $(W^S_t, W^v_t)$ driving the Heston dynamics, and $\sigma_{v,r}$ controls regime-dependent dispersion. We set $\mu_v = 0$ throughout all experiments. This construction ensures that any predictive relationship between volume and execution costs reflects the engineered semantic inversion (\Cref{eq:lambda_inversion}) rather than incidental correlation with the underlying stochastic volatility state.


To isolate semantic inversion from regime leakage via distributional moments, our primary identification control uses variance-matched volume (\Cref{subsec:variance_matched}). As a robustness check, we also evaluate an overlap-conditioned stress test (OCST) that restricts Regime~1 episodes to the Regime~0 training support of $\log \mathrm{Vol}_t$.

\subsection{Execution Costs and Semantic Inversion}
\label{subsec:inversion}
Trading costs follow the quadratic-impact specification in \cref{eq:cost}. Concretely, with position
$a_t$ and trade $\Delta a_t=a_t-a_{t-1}$, we use
\[
C_{r,t}(\Delta a_t;X_t)
= c_{\mathrm{spread}}|\Delta a_t| + \frac{1}{2}\lambda_{r,t}(\mathrm{Vol}_t)\,(\Delta a_t)^2.
\]
To avoid pathological spikes when $\mathrm{Vol}_t$ is extremely small, we clip the activity proxy below by
$\varepsilon>0$ and define $\mathrm{Vol}_t^\varepsilon=\max(\mathrm{Vol}_t,\varepsilon)$ in the impact terms.

The regime-dependent impact coefficient is
\begin{equation}
\label{eq:lambda_inversion}
\lambda_{0,t}(\mathrm{Vol}_t) = \frac{\lambda_0}{\mathrm{Vol}^\varepsilon_t},
\qquad
\lambda_{1,t}(\mathrm{Vol}_t) = \lambda_1\,\mathrm{Vol}_t,
\end{equation}
so that $\partial \lambda_{0,t}/\partial \mathrm{Vol}_t<0$ while $\partial \lambda_{1,t}/\partial \mathrm{Vol}_t>0$,
satisfying \Cref{def:semantic_inversion}. Intuitively: ``high volume $\Rightarrow$ low impact'' in Regime 0,
but ``high volume $\Rightarrow$ high impact'' in Regime 1. This constitutes the core \emph{Semantic Inversion}.

\paragraph{P\&L and loss.}
We simulate a self-financing hedging portfolio with execution costs:
\[
V_{t+1}=V_t + a_t\,(S_{t+1}-S_t) - C_{r,t}(\Delta a_t;X_t),\qquad V_0=0,
\]
and define the terminal hedging loss as $Y_r^{\theta,\phi}=L - V_T$. Since $Y_r^{\theta,\phi}$ is measured in the
same currency units as $S_t$, we report and optimize \emph{normalized} losses $\widetilde{Y}_r^{\theta,\phi}
:= Y_r^{\theta,\phi}/S_0$ so that risk and regularization terms operate on stable, dimensionless scales.

\subsection{Variance-Matched Control and Structural Priors}
\label{subsec:variance_matched}
The Variance-Matched Control serves as our primary identification pillar. By explicitly standardizing the activity proxy noise to $\sigma_{v,0} = \sigma_{v,1} = 0.30$ across both regimes, we ensure that the activity proxy $\mathrm{Vol}_t$ is marginally identical across regimes. This configuration eliminates the ``distributional shift'' confounder and ensures that any performance degradation in the stress regime is purely attributable to \emph{Semantic Inversion} in the execution-cost mapping (\Cref{def:semantic_inversion}).

\paragraph{Why variance-matching works.}
Because $\mathrm{Vol}_t$ is independent lognormal with matched parameters across regimes, variance-matching makes $\mathrm{Vol}_t$ marginally identical under both regimes, removing regime leakage through moments; remaining differences arise only from the sign flip in $\lambda_r(\mathrm{Vol}_t)$ (\Cref{eq:lambda_inversion}).

To further harden the policy against directional instability, we implement \emph{Hierarchical Information Regularization}, a \emph{structural prior} that assigns different information budgets to signals based on their structural relationship to the hedging objective.

\paragraph{Economic justification.}
The distinction between payoff-anchored and equilibrium-derived signals is not merely terminological but reflects fundamentally different information-cost economics:
\begin{itemize}
 \item \textbf{Payoff-anchored signals} (Greeks) are defined by the contractual payoff structure. Their decision-theoretic optimality, the direction of the optimal hedge adjustment, does not depend on which equilibrium obtains. Greeks can be computed from the contract and public price data without regime inference.
 \item \textbf{Equilibrium-derived signals} (microstructure) are endogenous objects whose meaning depends on prevailing market conditions. Safely exploiting them requires inferring which equilibrium applies, a costly operation requiring monitoring, data fusion, and organizational coordination.
\end{itemize}
We implement this distinction through differential $\beta$ penalties: the Price-encoder channel receives a low penalty $\beta_{\text{price}} = 0.00001$ (anchoring the hedging backbone), while the Microstructure-encoder channel receives a higher $\beta_{\text{micro}}$ (making regime-contingent exploitation costly). This is not an arbitrary hyperparameter choice but a \emph{rational inattention hierarchy}: scarce information-processing capacity should not be spent on encoding regime-contingent logic unless the value of microstructure exploitation exceeds the cost of regime disambiguation.

\subsection{Representations}
\label{subsec:features}
We evaluate four representation sets that differ in the information available to the policy.

\begin{table}[H]
\centering
\caption{State representations used in experiments. All representations share the same base state.}
\label{tab:representations}
\renewcommand{\arraystretch}{1.15}
\begin{tabularx}{\textwidth}{l X X}
\toprule
\textbf{Representation} & \textbf{Included features} & \textbf{Description} \\
\midrule
Base &
$\tau_t,\ \log(S_t/K),\ a_{t-1}$ &
Time to maturity, moneyness, and previous hedge position \\

Greeks &
Base $+\ \Delta^{BS}_t,\ \Gamma^{BS}_t$ &
Black--Scholes delta and gamma \\

Microstructure &
Base $+\ \mathrm{Vol}_t,\ r^{(1)}_t,\ \hat{\sigma}^{(5)}_t$ &
Observed activity and short-horizon return statistics \\

Combined &
Union of structural and microstructure features &
Full information set \\
\bottomrule
\end{tabularx}
\end{table}


Here $\tau_t$ denotes time to maturity, $r^{(1)}_t=\log(S_t/S_{t-1})$ is the most recent observed return
(defined for $t\ge1$, set to $0$ at $t=0$), and $\hat{\sigma}^{(5)}_t$ is a rolling realized-volatility
estimator computed from the last five returns,
$\hat{\sigma}^{(5)}_t=\sqrt{\frac{1}{5}\sum_{i=0}^{4}(r^{(1)}_{t-i})^2}$,
with truncation for $t<5$.


\subsection{Neural Network Architecture}
\label{subsec:arch}
We use a VIB encoder plus a policy head. The latent dimension is 8.
\noindent\emph{Remark.} The latent dimension is 8, chosen to balance expressiveness and compression. In preliminary tests (not shown), dimensions in $[4,16]$ yielded similar qualitative results.

\begin{table}[H]
\centering
\caption{Network architecture used in all experiments.}
\label{tab:arch}
\renewcommand{\arraystretch}{1.15}
\begin{tabularx}{\textwidth}{l >{\raggedright\arraybackslash}X >{\raggedright\arraybackslash}X}
\toprule
\textbf{Component} & \textbf{Architecture} & \textbf{Output} \\
\midrule
\textbf{Encoder} $q_\phi(z\mid X_t)$ & & \\
\quad Hidden layers & Dense(32) + ReLU $\rightarrow$ Dense(16) + ReLU & -- \\
\quad Output heads & Dense(8) [mean] and Dense(8) [log-std] & $\mu_\phi(X_t),\log\sigma_\phi(X_t)\in\mathbb{R}^8$ \\
\quad Sampling & $Z_t=\mu_\phi(X_t)+\sigma_\phi(X_t)\odot\epsilon$, $\epsilon\sim\mathcal{N}(0,I_8)$ & $Z_t\in\mathbb{R}^8$ \\
\addlinespace[0.35em]
\textbf{Policy} $\pi_\theta(Z_t)$ & & \\
\quad Hidden layer & Dense(16) + ReLU & -- \\
\quad Output & Dense(1) + $\tanh$, scaled by 2 & $a_t\in[-2,2]$ \\
\bottomrule
\end{tabularx}
\end{table}

\subsection{Training Protocol and Baselines}
\label{subsec:training}
We train on a balanced mixture of Regime~0 and Regime~1 episodes (50\% each), with the regime label $R$ withheld from the agent. This creates an identification challenge: on observations where optimal responses differ across regimes (the Conflict Set), the agent must either encode regime-contingent logic or compromise. Let $\mathcal{R}_r(\theta,\phi_p,\phi_m)=\mathrm{ES}_{0.95}(\widetilde{Y}_r^{\theta,\phi_p,\phi_m})$ denote the regime-$r$ risk of the \emph{normalized} terminal loss. We minimize the mixture-CVaR objective with hierarchical penalties:
\begin{equation}
\label{eq:train_obj}
\min_{\theta,\phi_p,\phi_m,\eta}\ \ 
\left\{\eta + \frac{1}{1-\gamma}\,\widehat{\mathbb E}\big[(\widetilde{Y}^{\theta,\phi_p,\phi_m}_{\mathrm{mix}}-\eta)_+\big]\right\}
\;+\; \beta_{\mathrm{price}}\,\widehat{\mathcal{C}}_{\mathrm{price}}(\phi_p)
\;+\; \beta_{\mathrm{micro}}\,\widehat{\mathcal{C}}_{\mathrm{micro}}(\phi_m),
\end{equation}
where $\widetilde{Y}^{\theta,\phi_p,\phi_m}_{\mathrm{mix}}$ is the normalized hedging error under the mixture distribution induced by $R\sim\mathrm{Bernoulli}(0.5)$ per episode, and $\widehat{\mathcal{C}}_{\mathrm{price}}, \widehat{\mathcal{C}}_{\mathrm{micro}}$ estimate the per-channel KL costs (\cref{eq:occams_objective}) in nats. Throughout the main sweep, we fix $\beta_{\mathrm{price}}=0.00001$ and vary $\beta_{\mathrm{micro}} \in \{0, 0.1, 0.5, 1.0, 5.0\}$.

\paragraph{Training details.}
\begin{table}[H]
\centering
\caption{Training configuration (see \Cref{tab:protocol} for complete protocol).}
\label{tab:training_details}
\small
\begin{tabular}{ll}
\toprule
Optimizer & Adam, lr $5 \times 10^{-3}$, batch 256 \\
Schedule & 200 epochs, 40-epoch warmup, 20 seeds \\
\bottomrule
\end{tabular}
\end{table}

\begin{table}[H]
\centering
\caption{Experimental Protocol Summary. This box consolidates the canonical design for reproducibility.}
\label{tab:protocol}
\renewcommand{\arraystretch}{1.15}
\begin{tabularx}{\textwidth}{l X}
\toprule
\textbf{Component} & \textbf{Specification} \\
\midrule
Regime assignment & $R \sim \mathrm{Bernoulli}(0.5)$ per episode, held constant within episode \\
Agent observability & Observes $X_t$; does \emph{not} observe regime label $R$ \\
Training distribution & 50\% Regime 0, 50\% Regime 1 episodes (mixed-regime) \\
Evaluation & Separate held-out test sets for $R=0$ and $R=1$; 10{,}000 paths each \\
Randomization per seed & Initial conditions, Brownian paths, regime draws \\
Fixed across seeds & Architecture, optimizer hyperparameters, $\beta$ values \\
\bottomrule
\end{tabularx}
\end{table}

\paragraph{Why this protocol identifies the mechanism.}
Mixed-regime training without regime labels creates a Conflict Set identification problem (\Cref{def:semantic_inversion}). The VIB penalty prices regime-contingent logic per the mechanism in \Cref{subsec:mechanism}. By varying $\beta$ and measuring ($R_0$, $R_1$, Deg), we trace the information--robustness frontier.

We choose the $\beta$ range on a logarithmic scale to span the transition from \emph{negligible compression} (average KL close to its unregularized level) to \emph{near-posterior collapse} (average KL sharply reduced). We report the realized information cost $\widehat{\mathcal{C}}(\phi)$ alongside risk to make the effective degree of information truncation explicit. In our calibration, the steep transition typically occurs between $\beta=10^{-1}$ and $\beta=1$.

\paragraph{Baselines.}
We compare:
\begin{enumerate}
\item \textbf{ERM (no info penalty):} $\beta=0$.
\item \textbf{$L_2$ weight decay:} ERM + $\lambda\|\theta\|_2^2$, $\lambda\in\{10^{-4},10^{-3},10^{-2}\}$.
\item \textbf{VIB (Occam's Hedge):} objective \cref{eq:train_obj}.
\item \textbf{Oracle regime (counterfactual):} same architecture as VIB and ERM, with the true regime label $r\in\{0,1\}$ appended to the input. To make this baseline informative, the oracle is trained on a balanced mixture of Regime 0 and Regime 1 episodes (still using the same loss and optimizer), and evaluated with the correct regime label provided at test time. This isolates whether failures under stress are primarily due to missing regime disambiguation (information) rather than insufficient network capacity.
\end{enumerate}

\subsection{Parameters and Calibration}

We summarize the simulator parameters used across all experiments in Table~\ref{tab:params}.

\label{subsec:params}
\begin{table}[t]
\centering
\caption{Simulator parameters used in experiments. Impact scales $(\lambda_0,\lambda_1)$ and spread cost $c_{\mathrm{spread}}$ are chosen so that execution costs are economically material and semantic inversion produces measurable stress degradation under Regime~1, while the variance-matched control removes distributional regime leakage in $\mathrm{Vol}_t$.}
\label{tab:params}
\renewcommand{\arraystretch}{1.15}
\begin{tabularx}{\textwidth}{l c >{\raggedright\arraybackslash}X}
\toprule
Parameter & Value & Description \\
\midrule
$S_0, K$ & 100 & Initial stock price and strike price \\
$\mu$ & 0.00 & Annualized drift (risk-neutral proxy) \\
$\kappa, \theta, \xi, \rho$ & 2.0, 0.04, 0.60, -0.60 & Heston parameters (speed, mean-var, vol-of-vol, corr) \\
$T$ & 30 & Steps to maturity (days) \\
$\Delta t$ & $1/252$ & Daily time step \\
$c_{\mathrm{spread}}$ & 0.001 & Proportional spread cost \\
$\lambda_0, \lambda_1$ & 0.5, 0.3 & Impact scale (Regime 0, 1) \\
$\mu_v$ & 0.0 & Volume log-mean (held constant across regimes) \\
$\sigma_{v}$ & 0.30 & Volume log-std (variance matched) \\
$\varepsilon$ & $10^{-3}$ & Volume floor in $\lambda_{0,t}$ \\
\bottomrule
\end{tabularx}
\end{table}

\subsection{Implementation Details and Design Choices}
\label{subsec:implementation_details}

\paragraph{Limited memory and the snapshot bottleneck.}
Our representation $Z_t$ conditions on $X_t$, which includes one lag of microstructure features ($K=1$). This limited memory prevents trivial regime identification from trajectory patterns while allowing short-term momentum signals. Combined with within-episode regime persistence (\Cref{ass:timescales}), reliable regime inference would require extended temporal integration, which incurs sustained information cost under the VIB penalty. We do not model explicit memory costs; the assumption is that single-snapshot regime identification is unreliable by design.

\paragraph{Hierarchical penalty implementation.}
We implement hierarchical regularization through separate encoding channels for price and Greeks (receiving $\beta_{\text{price}}=0.00001$) and microstructure features (receiving $\beta_{\text{micro}}$ from the sweep). Each channel has its own encoder $q_\phi^{(\cdot)}(z^{(\cdot)} \mid x^{(\cdot)})$, and the per-channel KL costs are summed with their respective $\beta$ weights. This structural prior reflects the economic distinction between payoff-anchored and equilibrium-derived signals (\Cref{subsec:variance_matched}). Additional implementation details (joint $\eta$ optimization and information cost variation) are provided in \Cref{app:impl_details}.

\FloatBarrier
\section{Experiments}
\label{sec:experiments}

This section specifies the evaluation protocol, reported metrics, and experimental tests used to validate the semantic-inversion mechanism and the information--robustness trade-off.

\subsection{Modeling Choices and Realism}
\label{subsec:realism}

Our simulator is purposefully stylized to isolate a specific failure mode, semantic inversion in microstructure signals, rather than replicate comprehensive market dynamics. We briefly connect key modeling choices to empirical microstructure literature and clarify external validity boundaries.

\paragraph{Volume--impact inversion.}
The sign flip in \cref{eq:lambda_inversion} is grounded in the distinction between liquidity-driven and capacity-constrained regimes. In normal markets, elevated trading activity often reflects greater market depth and lower marginal impact \citep{bouchaud2008digest}. Under stress, when balance sheets bind, elevated activity may instead reflect forced liquidation and adverse selection, \emph{increasing} marginal impact \citep{brunnermeier2009liquidity,duffie2020intermediation}. Our stylized specification isolates this channel by making the sign relationship explicit and controllable.


\paragraph{Episode-constant regime.}
Regime persistence (Assumption~\ref{ass:timescales}) simplifies identification by preventing trivial within-episode regime inference. In practice, regimes shift on horizons long relative to typical hedging frequency, making this a reasonable first-order approximation. Extensions with Markov switching are left to future work.

\paragraph{What is not modeled.}
We do not model endogenous market depth, limit order book dynamics, inventory risk, or strategic interaction among market participants. These omissions are deliberate: our goal is mechanism isolation, not calibration accuracy. The stylization enables clean identification of the information--robustness trade-off, which can then be validated in richer settings.


\subsection{Evaluation Metrics}
\label{subsec:exp_metrics}

Training uses the mixed-regime protocol (\Cref{tab:protocol}); the oracle counterfactual (\Cref{subsec:exp_baselines}) receives $R$. We evaluate on held-out episodes from both regimes. Let $\widetilde{Y}_r$ denote the normalized terminal loss in regime $r$ (as defined in \Cref{subsec:inversion}). We report tail risk via
\[
R_0=\mathrm{ES}_{0.95}(\widetilde{Y}_0),\qquad
R_1=\mathrm{ES}_{0.95}(\widetilde{Y}_1),\qquad
\mathrm{Deg}=\frac{R_1}{R_0},
\]
together with the realized information cost
\[
\mathcal{C}(\phi)=\mathbb{E}\!\left[\frac{1}{T}\sum_{t=0}^{T-1}\mathrm{KL}\!\left(q_\phi(Z_t\mid X_t)\,\|\,p(Z_t)\right)\right],
\]
measured in nats on the Regime~0 data used for training and validation. Lower is better for $R_0$, $R_1$, and $\mathrm{Deg}$. We report $\mathcal{C}(\phi)$ for interpretation rather than as a performance metric.

\paragraph{Trading diagnostics.}
To isolate whether performance changes are driven by trading intensity versus execution-cost exposure, we also report average turnover and average execution costs:
\[
\mathrm{Turnover}_r=\mathbb{E}\!\left[\sum_{t=0}^{T-1}|\Delta a_t|\;\middle|\;R=r\right],\qquad
\mathrm{Cost}_r=\mathbb{E}\!\left[\sum_{t=0}^{T-1} C_{r,t}(\Delta a_t;X_t)\;\middle|\;R=r\right].
\]
All results are reported as mean $\pm$ standard deviation over 20 random seeds.

\begin{table}[t]
\centering
\caption{Summary results across representations (mean $\pm$ std over 20 seeds). Lower is better for $R_0$, $R_1$, and Deg. The ERM-trained combined model achieves the lowest baseline risk but exhibits cross-regime fragility. VIB regularization at $\beta=0.5$ substantially reduces degradation. The Oracle row is a single-run counterfactual diagnostic (not averaged over seeds) showing that supplying the regime label eliminates stress fragility; ``N/A'' indicates the information cost is not applicable since regime labels bypass the compression channel.}
\label{tab:main_results}
\begin{tabular}{lcccc}
\toprule
Policy & $R_0$ (Regime 0) & $R_1$ (Regime 1) & Deg $=R_1/R_0$ & $\mathcal{C}(\phi)$ \\
\midrule
Greeks-only (ERM) & $0.194 \pm 0.004$ & $0.226 \pm 0.021$ & $1.17$ & $51.6 \pm 9.6$ \\
Micro-only (ERM) & $0.194 \pm 0.006$ & $0.222 \pm 0.013$ & $1.15$ & $99.2 \pm 19.6$ \\
Combined (ERM, $\beta=0$) & $0.194 \pm 0.004$ & $0.225 \pm 0.019$ & $1.16$ & $90.0 \pm 22.3$ \\
Combined (VIB, $\beta=0.5$) & $0.279 \pm 0.028$ & $0.294 \pm 0.020$ & $1.05$ & $0.47 \pm 0.36$ \\
\midrule
Oracle (regime label, diagnostic) & $0.252$ & $0.243$ & $0.97$ & N/A \\
\bottomrule
\end{tabular}
\end{table}

\subsection{Experimental Tests}
\label{subsec:exp_tests}

\subsubsection{Directional-Stability Verification}
\label{subsec:exp_stability}

We verify that the constructed microstructure channel satisfies \emph{semantic inversion in the sense of \Cref{def:semantic_inversion}}. In the simulator, inversion is implemented through the cost elasticity of the impact coefficient $\lambda_{r,t}(\cdot)$. For \cref{eq:lambda_inversion}, we define
\[
\mathrm{CE}_{\mathrm{Vol}}
:=
\mathrm{sign}\!\left(\frac{\partial \lambda_{0,t}}{\partial \mathrm{Vol}_t}\right)\cdot
\mathrm{sign}\!\left(\frac{\partial \lambda_{1,t}}{\partial \mathrm{Vol}_t}\right)
=-1.
\]
By \Cref{lem:damped_adjustment}, this cost-elasticity inversion implies a corresponding inversion in the optimal aggressiveness response to $\mathrm{Vol}_t$ on any overlap region where the frictionless adjustment is non-degenerate.


\begin{table}[t]
\centering
\caption{Semantic-inversion checklist (template). Only the volume channel is constructed to invert via the sign flip of
$\partial \lambda_{r,t}/\partial \mathrm{Vol}_t$. Other inputs are not hard-coded to invert and serve as controls.}
\label{tab:ds_map}
\begin{tabular}{lcc}
\toprule
Feature family & Coordinates & Inversion constructed? \\
\midrule
Volume proxy & $\mathrm{Vol}_t$ & Yes ($\mathrm{CE}_{\mathrm{Vol}}=-1$) \\
Greeks & $\Delta^{BS}_t,\Gamma^{BS}_t$ & No \\
Return statistics & $r^{(1)}_t,\hat{\sigma}^{(5)}_t$ & No \\
Base state & $\tau_t,\log(S_t/K),a_{t-1}$ & No \\
\bottomrule
\end{tabular}
\end{table}

\subsubsection{Latent Fragility and Stress Degradation}
\label{subsec:exp_latent_fragility}

We evaluate whether microstructure inputs create latent fragility: comparable in-regime performance (Regime~0) yet degradation under semantic inversion (Regime~1). Concretely, we train ERM policies ($\beta_{\mathrm{micro}}=0$) on the mixed-regime training distribution (50\% Regime~0, 50\% Regime~1 episodes, regime label withheld) for the Greeks-only, micro-only, and combined representations, and evaluate on held-out Regime~0 and Regime~1 test sets separately. We report $(R_0,R_1,\mathrm{Deg})$ and trading diagnostics. This test establishes (i) baseline indistinguishability under variance-matching and (ii) regime-shift fragility induced by semantic inversion.

\subsubsection{Information--Robustness Frontier (VIB Sweep)}
\label{subsec:exp_frontier}

For the combined representation, we sweep the information-price parameter
\[
\beta \in \{0,0.1,0.5,1.0,5.0\}
\]
and record $(R_0,R_1,\mathrm{Deg},\mathcal{C}(\phi))$. The resulting points define an \emph{information--robustness frontier}, quantifying the trade-off between baseline efficiency ($R_0$) and out-of-regime robustness ($R_1$) as the representation bandwidth is tightened.

\subsubsection{Mechanism Diagnostics: Residual Regime Information and Oracle Disambiguation}
\label{subsec:exp_diagnostics}

\noindent\emph{Identification controls.} Our primary identification control is the variance-matched volume environment (\Cref{subsec:variance_matched}); we report key performance results under this control. As a robustness check, we additionally report results under the overlap-conditioned stress test (OCST) described in \Cref{subsec:vol_proxy}.

We report two diagnostics aligned with the mechanism in \Cref{subsec:mechanism}.

\paragraph{Regime-probe accuracy.}
For each trained VIB model (each $\beta$), we measure residual regime information in the representation by training a lightweight probe classifier to predict the latent regime $R$ from time-averaged latent codes $\bar Z=\frac{1}{T}\sum_{t=0}^{T-1} Z_t$ on a balanced dataset of Regime~0/1 episodes. We report probe accuracy (and optionally an estimated $I(R;Z)$ if computed) as a direct measurement of residual regime information after paying the information price.


\paragraph{Oracle-regime counterfactual.}
To distinguish informational failure from function-approximation failure, we train an oracle policy that receives the true regime label appended to inputs. The oracle is trained on a balanced mixture of Regime~0 and Regime~1 episodes and evaluated with the correct label at test time. A large improvement in Regime~1 outcomes for micro and combined inputs under oracle access constitutes direct evidence that stress fragility is primarily due to missing regime-disambiguating information rather than intrinsic uselessness of microstructure features.

\subsection{Baselines and Model Selection}
\label{subsec:exp_baselines}

We compare: (i) ERM ($\beta=0$), (ii) $L_2$ weight decay on the combined representation with $\lambda\in\{10^{-4},10^{-3},10^{-2}\}$, (iii) VIB and Occam's Hedge with the $\beta$ sweep above, and (iv) the oracle-regime counterfactual. When a single ``best'' model is required for summary comparisons, we select $\lambda$ or $\beta$ by Regime~0 validation performance (early stopping criterion), and then report the corresponding out-of-regime metrics on Regime~1.

\section{Results}
\label{sec:results}

This section reports empirical evidence for the paper's central mechanism: (i) \emph{latent fragility} in which microstructure features achieve comparable in-sample performance under stable semantics but create hidden vulnerability, (ii) cross-regime degradation under semantic inversion for ERM-trained policies, and (iii) an \emph{information--robustness trade-off} under VIB regularization. 

All experiments use a controlled Heston-based simulator with a minimal construction of semantic inversion in the volume--impact channel (\Cref{subsec:inversion}). We simulate 20 independent training seeds, each with mixed-regime training (50\% Regime~0, 50\% Regime~1 episodes). Unless otherwise stated, all reported values are mean $\pm$ standard deviation over these 20 seeds, evaluated on held-out test episodes.

\subsection{Baseline Performance and Latent Fragility (Regime 0)}
\label{subsec:res_latent_fragility}

\textbf{Hypothesis.} Under the variance-matched volume control (\Cref{subsec:variance_matched}), all representations, Greeks-only, micro-only, and combined, should achieve comparable in-regime performance $R_0$. The latent fragility (\Cref{def:latent_fragility}) in this identification design is not baseline outperformance, but \emph{latent wrong-way exposure}: microstructure-heavy representations implicitly encode volume--impact correlations that become harmful under regime shift.

\textbf{Protocol.} Train ERM policies ($\beta=0$) on mixed-regime training data for each representation set, and evaluate $R_0=\mathrm{ES}_{0.95}(\widetilde{Y}_0)$ on held-out Regime~0 episodes.

\begin{table}[t]
\centering
\caption{Baseline (Regime 0) performance under ERM ($\beta=0$) with variance-matched volume control. Values are $\mathrm{ES}_{0.95}$ of normalized terminal loss, mean $\pm$ std over 20 seeds. Under variance-matching, all representations achieve comparable in-regime performance, by design. The trap is wrong-way exposure under stress (\Cref{tab:wrong_way}), not baseline outperformance.}
\label{tab:baseline_comp}
\begin{tabular}{lcc}
\toprule
Representation (ERM) & $R_0$ & Diff.\ vs.\ Greeks \\
\midrule
Greeks-only & $0.194 \pm 0.004$ & -- \\
Micro-only & $0.194 \pm 0.006$ & $+0.0\%$ \\
Combined & $0.194 \pm 0.004$ & $+0.0\%$ \\
\bottomrule
\end{tabular}
\end{table}

\paragraph{Interpretation.}
Under the variance-matched control, all three representations achieve indistinguishable baseline performance ($R_0 \approx 0.194$) \emph{and} near-identical stress degradation at $\beta=0$: Greeks-only Deg$=$1.17, Micro-only Deg$=$1.15, Combined Deg$=$1.16. This uniformity is \emph{by design}: variance-matching eliminates distributional leakage ($\sigma_{v,0} = \sigma_{v,1}$), so any microstructure advantage would require exploiting its \emph{semantic} relationship to execution costs, which is regime-dependent and cannot be safely exploited without regime labels.

The shared baseline degradation (Deg$\approx$1.15--1.17) across all representations reflects the BS-Heston model mismatch: Greeks computed via Black-Scholes assume constant volatility, while the simulator uses stochastic volatility. Crucially, microstructure features provide \emph{neither benefit nor additional harm} at $\beta=0$ under variance-matching, because the regime information required for safe exploitation is unavailable from single snapshots ($I(R;X_t) \approx 0$). 


\paragraph{Reconciling ES-based metrics with trading diagnostics.}
Under variance matching, ES-based metrics ($R_0$, $R_1$, Deg) can be dominated by BS--Heston model mismatch, which affects all representations similarly. As a result, microstructure may not materially move ES at $\beta=0$. However, wrong-way exposure $W_1 > 0$ remains detectable in trading diagnostics even when aggregate risk looks comparable. We therefore report $W_1$ as a mechanistic diagnostic that isolates directional errors from model-mismatch confounds.
The latent fragility is \emph{not} visible in differential ERM degradation---it is visible in the oracle test (\Cref{tab:main_results}), which shows that providing regime labels unlocks the full value of microstructure (Deg$=$0.97). This is the governance failure: a practitioner selecting models by in-regime validation would be indifferent among these representations, unaware that microstructure requires regime disambiguation to exploit safely. Only the oracle counterfactual reveals that microstructure is valuable when the required regime-disambiguating information is supplied exogenously.


\subsection{Stress Fragility Under Semantic Inversion (Regime 1)}
\label{subsec:res_fragility}

\textbf{Hypothesis.} Under variance-matching, all representations should exhibit similar stress degradation at $\beta=0$ because regime information is unavailable from single snapshots. The shared degradation (Deg $\approx 1.16$) reflects BS-Heston model mismatch, not differential microstructure fragility. The VIB sweep should reduce degradation by suppressing dependence on semantically unstable signals.

\textbf{Protocol.} Evaluate the Regime~0-trained ERM policies on Regime~1 episodes without retraining. Report $R_1=\mathrm{ES}_{0.95}(\widetilde{Y}_1)$, degradation $\mathrm{Deg}$, and trading diagnostics (turnover and execution costs).

\begin{figure}[t]
\centering
\includegraphics[width=0.8\linewidth]{figures/fig_frontier_beta_sweep.png}
\caption{Information--Robustness Frontier (Combined representation, 20 seeds). The plot demonstrates the trade-off between baseline efficiency ($R_0$) and stress robustness ($R_1$) as the information price $\beta$ increases. At $\beta=0$ (ERM), the policy achieves low baseline risk but exhibits substantial stress degradation. At $\beta \approx 0.5$, the policy achieves near-parity between regimes, representing the robustness sweet spot. At higher $\beta$, both risks converge toward the Greeks-only baseline.}
\label{fig:occam_frontier}
\end{figure}

\begin{figure}[t]
\centering
\includegraphics[width=0.75\linewidth]{figures/fig_semantic_flip_correlations.png}
\caption{Semantic Inversion Verification (20 seeds). The correlation between the volume proxy $V_t$ and the impact coefficient $\lambda_t$ exhibits a clear sign flip across regimes: negative in Regime 0 (high volume $\Rightarrow$ low impact) and positive in Regime 1 (high volume $\Rightarrow$ high impact). This confirms that the constructed inversion is statistically stable across all seeds.}
\label{fig:semantic_flip}
\end{figure}

\begin{figure}[t]
\centering
\includegraphics[width=0.48\linewidth]{figures/fig_thermomap_beta_0.0.png}
\includegraphics[width=0.48\linewidth]{figures/fig_thermomap_beta_0.5.png}
\caption{Policy Surface Heatmaps (Thermomaps) for $\beta=0$ (left) and $\beta=0.5$ (right). At $\beta=0$, the policy surface is highly sensitive to the Volume proxy $V_t$ (vertical axis), exhibiting the "ragged" exploitation of the Regime 0 correlation. At $\beta=0.5$, the information bottleneck collapses the vertical dependence, resulting in a "smooth" policy surface that depends primarily on Price ($S_t$), effectively recovering a robust Greek-like hedge that is invariant to the semantically unstable volume signal.}
\label{fig:thermomaps}
\end{figure}

\paragraph{Interpretation.}
Under variance-matching, all representations exhibit similar degradation (Deg $\approx 1.15$--$1.17$), confirming that microstructure provides neither benefit nor additional harm at $\beta=0$ when regime information is unavailable. The stress fragility is \emph{not} visible in differential ERM degradation---it is visible in the VIB sweep (\Cref{subsec:res_tradeoff}) and the oracle test (\Cref{tab:main_results}), which reveal the information--robustness trade-off and confirm the failure is informational.

\subsection{Information--Robustness Trade-off (VIB Sweep)}
\label{subsec:res_tradeoff}

\textbf{Hypothesis.} Increasing the information price $\beta$ reduces representational bandwidth (lower realized $\mathcal{C}(\phi)$), which should (i) mildly worsen baseline performance $R_0$ by limiting exploitation of microstructure shortcuts, but (ii) improve stress performance $R_1$ and reduce degradation by suppressing regime-contingent logic that cannot be safely implemented under inversion. Models with smaller $\mathcal{C}(\phi)$ should lie closer to the ``no-breakdown'' diagonal $R_1\approx R_0$.

\textbf{Protocol.} Using the combined representation with lagged microstructure features ($K=1$ lag) and previous-action memory, we sweep $\beta\in\{0,0.1,0.5,1.0,5.0\}$. For each $\beta$, we report baseline risk $R_0=\mathrm{ES}_{0.95}(\widetilde{Y}_0)$, stress risk $R_1=\mathrm{ES}_{0.95}(\widetilde{Y}_1)$, degradation ratio $\mathrm{Deg}=R_1/R_0$, and the realized information cost $\mathcal{C}(\phi)$ (in nats).

\begin{table}[t]
\centering
\caption{Information--Robustness Trade-off (Combined representation, 20 seeds). As $\beta$ increases, baseline efficiency degrades modestly while stress robustness improves substantially. The transition band $\beta\in[0.1,0.5]$ represents the steepest portion of the Occam Frontier.}
\label{tab:vib_sweep}
\begin{tabular}{lcccc}
\toprule
$\beta$ & $R_0$ & $R_1$ & Deg & $\mathcal{C}(\phi)$ \\
\midrule
0.0 (ERM) & $0.194 \pm 0.004$ & $0.225 \pm 0.019$ & $1.16$ & $90.0 \pm 22.3$ \\
0.1 & $0.212 \pm 0.007$ & $0.260 \pm 0.017$ & $1.23$ & $2.93 \pm 0.27$ \\
0.5 & $0.279 \pm 0.028$ & $0.294 \pm 0.020$ & $1.05$ & $0.47 \pm 0.36$ \\
1.0 & $0.311 \pm 0.007$ & $0.311 \pm 0.016$ & $1.00$ & $0.003 \pm 0.007$ \\
5.0 & $0.311 \pm 0.007$ & $0.310 \pm 0.016$ & $1.00$ & $0.0001 \pm 0.0001$ \\
\bottomrule
\end{tabular}
\end{table}

\paragraph{Empirical observations.}
The results exhibit a sharp information--robustness transition. At $\beta=0$ (ERM), the policy achieves low baseline risk but exhibits 16\% degradation under stress. Increasing $\beta$ to 0.5 substantially suppresses cross-regime fragility (degradation drops to 5\%), at the cost of a 44\% increase in baseline risk. At $\beta \ge 1.0$, the policy converges to a regime-invariant representation with degradation ratios statistically indistinguishable from unity, consistent with near-complete information bottleneck collapse.

The realized information cost $\mathcal{C}(\phi)$ decreases by over two orders of magnitude across the sweep, confirming that the KL penalty actively constrains representational bandwidth. The transition is concentrated in the interval $\beta\in[0.1,0.5]$, suggesting a critical information threshold below which regime-contingent logic becomes uneconomical.
At $\beta=0.1$, partial compression can weaken hedging responsiveness without fully suppressing dependence on semantically unstable signals; robustness improves once $\beta$ is large enough to effectively collapse that channel.



\subsection{Semantic Inversion Across Regimes}
\label{subsec:semantic_inversion_results}

\textbf{Identification strategy.} The core premise of \Cref{def:semantic_inversion} is that the mapping from the volume proxy $V_t$ to the marginal impact coefficient $\lambda_{r,t}$ exhibits a sign flip across regimes. To verify that this inversion materialized in the simulated data, we compute the correlation between $V_t$ and $\lambda_{r,t}$ separately for Regime~0 and Regime~1 episodes.

\textbf{Expected pattern.} By construction (\Cref{eq:lambda_inversion}), we have $\lambda_{0,t} \propto 1/V_t$ and $\lambda_{1,t} \propto V_t$. This implies
\[
\mathrm{Corr}(V_t, \lambda_t \mid R=0) < 0, \qquad \mathrm{Corr}(V_t, \lambda_t \mid R=1) > 0.
\]

\begin{table}[t]
\centering
\caption{Semantic inversion in the simulator (20 seeds). The sign of the Volume--Impact correlation flips across regimes, confirming that the constructed inversion is statistically stable and not an artifact of a single lucky seed.}
\label{tab:semantic_flip}
\begin{tabular}{lc}
\toprule
Regime & $\mathrm{Corr}(V_t, \lambda_t)$ \\
\midrule
Regime 0 (Baseline) & $-0.477 \pm 0.002$ \\
Regime 1 (Stress) & $+0.621 \pm 0.034$ \\
\bottomrule
\end{tabular}
\end{table}

\paragraph{Interpretation.}
The sign flip is consistent and stable across all 20 seeds, with tight standard deviations relative to the mean magnitudes. This confirms that the simulator successfully induces a Lucas-critique vulnerability in the volume--impact channel: an observable that is semantically informative in Regime~0 (high volume $\Rightarrow$ low marginal costs) reverses its economic meaning in Regime~1 (high volume $\Rightarrow$ high marginal costs). 

This inversion is not a numerical artifact but a structural feature of the equilibrium shift (\Cref{subsec:inversion}): in Regime~0, elevated trading activity reflects ample liquidity, whereas in Regime~1, it signals binding intermediation constraints. Consequently, a policy that internalizes the Regime~0 correlation without encoding regime-contingent logic will systematically trade in the wrong direction under stress.

\subsection{Posterior Collapse under Extreme Compression}
\label{subsec:posterior_collapse}

\paragraph{Cautionary note.}
For sufficiently large information prices, VIB-style objectives can induce near-complete collapse of the latent channel, yielding policies that underreact to payoff risk. In pilot checks with larger $\beta_{\mathrm{micro}}$ values beyond our main sweep (not reported in tables), we observed this failure mode: the model outputs near-constant positions, effectively abandoning the hedging mandate to minimize information cost.

Hierarchical penalties mitigate this failure mode by decoupling information budgets. By keeping payoff-anchored information essentially uncompressed ($\beta_{\mathrm{price}} \approx 0$) while pricing microstructure bandwidth via $\beta_{\mathrm{micro}}$, we ensure the policy retains the high-bandwidth information required for delta hedging. This preserves the hedging mandate while still suppressing regime-contingent microstructure logic.


\subsection{Mechanism Diagnostics}
\label{subsec:exp_diag}

We report two diagnostics aligned with the mechanism in \Cref{subsec:mechanism}.

\paragraph{Residual regime information in the representation.}
\textbf{Hypothesis.} In our variance-matched snapshot design, regime information is unavailable from single observations by construction ($I(R;X_t) \approx 0$). Therefore, regime-probe accuracy should remain near chance ($\approx 0.5$) across all $\beta$ values, confirming that the identification design blocks regime leakage. The variation in wrong-way exposure with $\beta$ (\Cref{tab:wrong_way}) reflects changes in dependence on semantically unstable signals, not changes in regime-information extraction.

\textbf{Measurement.} For each trained VIB model (each $\beta$), construct a probe dataset by simulating a balanced mixture of Regime~0 and Regime~1 episodes. For each episode, form an episode-level summary of the latent trajectory, e.g.,
\[
\bar Z := \frac{1}{T}\sum_{t=0}^{T-1} Z_t
\quad\text{or}\quad
\big(\bar \mu,\bar \sigma\big):=
\left(\frac{1}{T}\sum_{t=0}^{T-1}\mu_\phi(X_t),\ \frac{1}{T}\sum_{t=0}^{T-1}\sigma_\phi(X_t)\right),
\]
and train a simple classifier (logistic regression or a one-hidden-layer MLP) to predict $R$ from this summary using cross-validated accuracy on held-out probe data. We report probe accuracy and, optionally, an information estimate using the standard binary lower bound
\[
I(R;\bar Z)\ \ge\ \log 2 - h(p_e),\qquad p_e:=1-\text{acc},
\]
where $h(\cdot)$ is the binary entropy.

\textbf{Observed pattern and interpretation.} Regime probe accuracy remains near chance (AUC $\approx 0.5$) across all $\beta$ levels, \emph{including $\beta=0$}. This confirms our identification design: under episode-constant regimes (\Cref{ass:timescales}) with variance-matched volume (\Cref{subsec:variance_matched}), single-snapshot features contain essentially zero regime information ($I(R;X_t) \approx 0$). By data processing inequality, $I(R;Z_t) \le I(R;X_t) \approx 0$ regardless of $\beta$.

\paragraph{Mechanism clarification.}
The VIB penalty does \emph{not} operate by reducing regime information, which was already unavailable by design. Instead, it operates by pricing representational bandwidth in a way that prevents policies from developing dependencies on semantically unstable microstructure signals that would require regime disambiguation to exploit safely. When regime information is unavailable ($I(R;X_t) \approx 0$), \Cref{prop:regime_tax} implies wrong-way behavior is unavoidable on conflict states without additional information sources. The information constraint prevents the encoder from learning representations that \emph{presume} regime information is available.

The mechanism is not ``VIB suppresses regime information'' (there was none to suppress), but rather ``VIB makes regime-contingent exploitation costly, rationally biasing toward payoff-anchored features that do not require regime disambiguation.'' This explains why wrong-way scores $W_1$ drop with $\beta$ (\Cref{tab:wrong_way}): higher $\beta$ discourages learning dependencies that presume access to regime information the observation space cannot provide.


\paragraph{Oracle disambiguation test (diagnostic evidence).}
\textbf{Hypothesis.} If stress fragility is fundamentally informational, then supplying the missing regime label should restore the value of semantically unstable microstructure features. Concretely, micro-only and combined representations should perform well in Regime~1 when trained and evaluated with oracle access to $R$, and the degradation ratio should shrink substantially.

\textbf{Measurement.} Append the true regime label $R$ (or a one-hot encoding) to the policy input and retrain the same architecture on a balanced mixture of Regime~0 and Regime~1 episodes, then evaluate with the correct $R$ provided at test time. A large reduction in $R_1$ and $\mathrm{Deg}$ relative to the non-oracle counterparts constitutes a diagnostic ``diagnostic evidence'': it indicates that the model class can use microstructure safely when the required regime-disambiguating information is provided exogenously.

\textbf{Result.} The oracle policy achieves $R_0 = 0.252$ and $R_1 = 0.243$, yielding $\mathrm{Deg} = 0.97$. This near-perfect stress robustness (compared to Deg $\approx 1.16$ for non-oracle policies at $\beta=0$) confirms that the capacity to hedge under semantic inversion is present in the model class, but is blocked by missing regime-disambiguating information. The diagnostic evidence is complete: \emph{providing the regime label eliminates stress fragility entirely}.

\subsection{Comparison to Standard Regularization}
\label{subsec:exp_reg}

\textbf{Hypothesis.} $L_2$ weight decay may reduce overfitting and mildly improve stress performance, but it does not explicitly price regime-contingent information. Therefore, at matched baseline performance ($R_0$), VIB should achieve a lower degradation ratio and lower regime-probe accuracy, consistent with selective suppression of regime-dependent logic rather than generic capacity shrinkage.

\textbf{Measurement.} For the combined representation, select the best $L_2$ coefficient $\lambda$ by validation $R_0$ (Regime~0) and select the best VIB model by validation $R_0$ (equivalently, among the $\beta$ sweep). Compare ERM, best-$L_2$, and best-VIB on $(R_0,R_1,\mathrm{Deg})$, and include the regime probe accuracy for the learned representation.

\begin{table}[t]
\centering
\caption{Regularization comparison (combined representation, 20 seeds). VIB at $\beta=0.5$ achieves a substantially lower degradation ratio than ERM, suggesting that information constraints selectively suppress regime-contingent logic rather than merely reducing capacity.}
\label{tab:reg_compare}
\begin{tabular}{lcccc}
\toprule
Method (Combined) & $R_0$ & $R_1$ & Deg & $\mathcal{C}(\phi)$ \\
\midrule
ERM ($\beta=0$) & $0.194 \pm 0.004$ & $0.225 \pm 0.019$ & $1.16$ & $90.0 \pm 22.3$ \\
VIB ($\beta=0.1$) & $0.212 \pm 0.007$ & $0.260 \pm 0.017$ & $1.23$ & $2.93 \pm 0.27$ \\
VIB ($\beta=0.5$) & $0.279 \pm 0.028$ & $0.294 \pm 0.020$ & $1.05$ & $0.47 \pm 0.36$ \\
\bottomrule
\end{tabular}
\end{table}

\paragraph{Interpretation.}
If $L_2$ improves $\mathrm{Deg}$ only marginally while VIB yields a larger reduction in $\mathrm{Deg}$ and a more pronounced decrease in wrong-way exposure, this supports the view that Occam's Hedge operates by discouraging dependence on signals that would require regime disambiguation to use safely, rather than by merely shrinking parameters.

\subsubsection{Is the effect specific to information regularization? Matched-$R_0$ comparisons}
\label{subsec:matched_r0}

To rigorously isolate targeted information pricing from generic capacity reduction, we compare VIB to $L_2$ weight decay at \emph{matched baseline risk} $R_0$. As shown in \Cref{tab:matched_r0}, $L_2$ regularization does not achieve the same robustness gains. Key finding: $L_2$ models show consistently higher Probe AUC (regime predictability) than VIB, confirming that $L_2$ does not suppress regime-information extraction as effectively as the information bottleneck, although both remain near chance levels in this snapshot setting.

\begin{table}[t]
\centering
\caption{Matched-$R_0$ comparison between $L_2$ and VIB (Combined representation). L2 regularization ($L_2$) fails to reduce stress risk ($R_1$) to the same extent as simple VIB, even at matched or strictly better baseline risk ($R_0$).}
\label{tab:matched_r0}
\begin{tabular}{lcccc}
\toprule
Method & Regularization & $R_0$ & $R_1$ & ProbeAcc \\
\midrule
$L_2$ & $\lambda=0$ & 0.189 & 0.418 & 0.509 \\
$L_2$ & $\lambda=0.001$ & 0.187 & 0.429 & 0.511 \\
$L_2$ & $\lambda=0.010$ & 0.168 & 0.353 & 0.511 \\
$L_2$ & $\lambda=0.100$ & 0.174 & 0.378 & 0.511 \\
$L_2$ & $\lambda=1.000$ & 0.172 & 0.333 & 0.511 \\
VIB & $\beta=0$ & 0.193 & 0.342 & 0.510 \\
\bottomrule
\end{tabular}
\end{table}

\subsection{Wrong-Way Exposure Diagnostics}
\label{subsec:res_wrong_way}

\textbf{Hypothesis.} Cross-regime fragility under semantic inversion should manifest as directional trading errors: the policy trades more aggressively when marginal execution costs are high (\Cref{def:wrong_way}). 

\textbf{Protocol.} For each trained VIB model (at $\beta_{\mathrm{micro}}$ values in the sweep), we compute the canonical quantile-based wrong-way score $W_r$ (\Cref{def:wrong_way}) using $q=0.2$ (top/bottom 20\% quantiles). For computational efficiency and interpretability, we also report a correlation-based proxy:
\[
W^{\mathrm{corr}}_r := \mathrm{Corr}(|\ \Delta a_t|, V_t \mid R=r) \cdot \mathrm{sgn}\!\left(\frac{\partial \lambda_r}{\partial V_t}\right),
\]
which is equivalent to the quantile-based formulation when the impact function $\lambda_r(V_t)$ is monotonic in $V_t$ (as in our construction). By construction, $W_r > 0$ (or $W^{\mathrm{corr}}_r > 0$) indicates wrong-way behavior (trading more when costs are higher), while $W_r < 0$ indicates right-way behavior. Table~\ref{tab:wrong_way} reports $W^{\mathrm{corr}}_1$ (Regime~1); results using the quantile-based Definition~\ref{def:wrong_way} are qualitatively identical and omitted for brevity.

\begin{table}[t]
\centering
\caption{Wrong-way exposure across $\beta$ sweep (Combined representation, Regime 1). Values are mean $\pm$ std over 20 seeds. At $\beta=0$, the model exhibits statistically significant wrong-way trading. At $\beta \ge 0.5$, wrong-way exposure becomes statistically indistinguishable from zero.}
\label{tab:wrong_way}
\begin{tabular}{lc}
\toprule
$\beta$ & $W_1$ (Regime 1) \\
\midrule
0.0 (ERM) & $+0.110 \pm 0.017$ \\
0.1 & $+0.042 \pm 0.007$ \\
0.5 & $+0.005 \pm 0.019$ \\
1.0 & $+0.031 \pm 0.006$ \\
5.0 & $+0.032 \pm 0.006$ \\
\bottomrule
\end{tabular}
\end{table}

\paragraph{Interpretation.}
The ERM policy ($\beta=0$) exhibits substantial wrong-way exposure ($W_1 = +0.110$), consistent with the mechanism in \Cref{prop:regime_tax}: the policy internalizes the Regime~0 correlation (high volume $\Rightarrow$ low impact) but fails to adapt when this relationship inverts in Regime~1. 

Increasing $\beta$ to 0.5 substantially suppresses this directional error, with $W_1$ dropping to $+0.005 \pm 0.019$, statistically indistinguishable from zero at the 20-seed level. Interestingly, at $\beta \ge 1.0$, wrong-way exposure increases modestly to $\approx+0.031$. This nonmonotonic pattern suggests that at very high $\beta$, the information bottleneck collapses the microstructure channel entirely (\Cref{fig:occam_frontier}), and the residual correlation reflects only the baseline Greeks-driven hedge, which may incidentally correlate with volume through the underlying volatility dynamics. Crucially, this residual correlation is not a re-emergence of regime-contingent logic, but an incidental artifact of payoff-driven hedging under stochastic volatility (e.g., Greeks--vol coupling in the Heston model).

\paragraph{Metric choice.}
We report the correlation-based wrong-way metric $W_r := \mathrm{Corr}(|\Delta a_t|, V_t \mid R=r) \cdot \mathrm{sgn}(\partial \lambda_r / \partial V_t)$ as our canonical measure in \Cref{tab:wrong_way}. This formulation is computationally stable and directly interpretable. The quantile-based formulation in \Cref{def:wrong_way} is theoretically more general (it handles non-monotonic cost schedules) but yields qualitatively similar results in our setting with monotonic regime-conditional impact functions. Both metrics agree on the key finding: wrong-way exposure drops substantially as $\beta$ increases from 0 to 0.5.

\subsection{Robustness Checks}
\label{subsec:robustness}

To confirm the mechanism is not an artifact of specific parameter choices, we report sensitivity checks along the following dimensions (results to be reported):
\begin{itemize}
\item \textbf{Wrong-way quantile sensitivity:} $q \in \{0.1, 0.2, 0.3\}$ in \Cref{def:wrong_way}.
\item \textbf{Impact scale sensitivity:} vary $(\lambda_0, \lambda_1)$ jointly while preserving the inversion sign.
\item \textbf{Variance-matched dispersion sensitivity:} vary $\sigma_v$ with $\sigma_{v,0} = \sigma_{v,1}$.
\item \textbf{Feature ablation:} remove return-stat features $(r^{(1)}_t, \hat{\sigma}^{(5)}_t)$ and confirm qualitative stability.
\end{itemize}
These checks establish that the information--robustness trade-off is not brittle to small perturbations in the experimental design.


\subsection*{Key Takeaways}
\begin{enumerate}
 \item \textbf{Latent fragility confirmed.} Under the variance-matched control, all representations achieve indistinguishable baseline performance ($R_0 \approx 0.194$) and near-identical stress degradation at $\beta=0$ (Deg $\approx 1.16$). The latent fragility is not visible in differential ERM degradation, but in the oracle test (Deg $= 0.97$), which confirms microstructure is valuable when regime-disambiguating information is provided. VIB at $\beta \ge 1.0$ achieves Deg $\approx 1.0$ by suppressing dependence on semantically unstable signals.
 
 \item \textbf{Wrong-way exposure drops with $\beta$.} The wrong-way score $W_1$ falls from $+0.11$ (ERM) to $\approx 0$ at $\beta=0.5$, confirming that the information penalty suppresses directionally unstable logic.
 
 \item \textbf{Trade-off is quantifiable.} The information--robustness frontier shows that eliminating stress fragility costs 60\% higher baseline ES, a meaningful but potentially acceptable price for portfolios with material regime-shift exposure.
\end{enumerate}

\section{Conclusion}
\label{sec:conclusion}
We document a regime-shift failure mode for deep hedging with execution costs: microstructure proxies can invert their marginal execution-cost meaning across latent regimes, causing ERM-trained policies to exhibit wrong-way trading under stress. We formalized semantic inversion as directional instability (\Cref{def:semantic_inversion}), distinguished it from magnitude instability (scale change with preserved sign), and proposed Occam's Hedge as an information-constrained policy class implemented via a VIB KL-to-prior penalty.

The proposed mechanism is conditional logic: safely using sign-flipping signals requires inferring the latent regime, which is directly priced under an information constraint (\Cref{prop:regime_tax}). In our controlled identification environment where regime information is unavailable from snapshots by design, we find: (i) near-identical baseline degradation across all representations (Deg $\approx 1.16$), confirming microstructure cannot be safely exploited without regime labels; (ii) an information--robustness frontier where VIB reduces degradation to 1.0 by suppressing dependence on semantically unstable signals; and (iii) an oracle disambiguation test (Deg $= 0.97$) confirming the failure is purely informational.


A central next step is empirical validation on real market data, particularly during stressed episodes such as March 2020, to test whether volume--spread sign instabilities arise in practice and to quantify the resulting hedging degradation. Second, deploying richer endogenous liquidity models (e.g., limit order book dynamics or inventory constraints) would clarify whether the mechanism generalizes beyond our stylized simulator. Third, developing practical diagnostics for detecting latent fragility in production systems before regime shifts occur would inform model-risk governance for learned trading policies.


\appendix

\section{Additional Implementation Details}\label{app:impl_details}

\paragraph{Joint optimization of $\eta$ in CVaR.}
We optimize the VaR threshold $\eta$ jointly with policy parameters $(\theta,\phi)$ via stochastic gradient descent using the Rockafellar--Uryasev formulation (\Cref{eq:cvar_ru}). This avoids explicit quantile estimation, which can be noisy in small batches, and provides stable gradients throughout training. In practice, $\eta$ converges to approximately the 95th percentile of the hedging error distribution under the current policy.

\paragraph{Information cost variation at $\beta=0$.}
\Cref{tab:main_results} shows different realized information costs $\mathcal{C}(\phi)$ across representations despite similar baseline risk $R_0$. At $\beta=0$, the KL penalty is inactive, so the encoder may extract information from inputs without cost. Representations with higher-dimensional inputs (Combined, Micro-only) naturally encode more information even when not penalized, explaining the variation. This is not a confound: the key comparison is how $\mathcal{C}(\phi)$ changes with $\beta$ within each representation.

\section{Assumptions and regularity conditions}\label{app:assumptions}

\begin{assumption}[Differential Regime Dependence]
\label{ass:hierarchy}
We assume that exploiting certain signals requires encoding regime-contingent distinctions, in the sense that their correct use depends on the latent regime $R_t$, whereas payoff-anchored hedging relationships can be implemented from the current observable state without explicit regime resolution. Consequently, under an information-processing constraint, signals whose usefulness depends on regime identification are disfavored relative to regime-robust risk controls.
\end{assumption}

\begin{assumption}[Separation of Time Scales]
\label{ass:timescales}
The latent regime $R\in\{0,1\}$ is persistent relative to the trading frequency. In particular, regime information cannot be reliably inferred from a single snapshot $X_t$ and would (in a richer extension) require temporal integration over histories. In our simulator, we implement this persistence by drawing $R$ once per episode and holding it fixed.
\end{assumption}

\begin{assumption}[Exogenous Cost Dynamics]
\label{ass:exogenous_costs}
While the agent's actions incur execution costs, we assume the functional form of the cost schedule (e.g., $\lambda_{r,t}(\cdot)$ or $C_{r,t}(\cdot)$) is driven by exogenous market state variables or aggregate market flow and is not permanently altered by the trading of an individual, small-scale agent. This rules out strategic feedback loops in which the agent optimizes by manipulating the regime itself.
\end{assumption}


\section{Proof of \Cref{prop:irreducible_ambiguity}}\label{app:proof-prop_ambiguity}

We first make explicit mild regularity conditions under which regime-optimal policies are
pointwise unique minimizers and the two regimes overlap on the conflict set.

\begin{assumption}[Pointwise strict optimality and overlap on $\mathcal X_{\mathrm{conflict}}$]\label{ass:strict-opt}
For each regime $r \in \{0,1\}$ and each observation $x \in \mathcal X$, the conditional
risk admits a pointwise representation
\[
\mathcal R_r(\pi) \;=\; \mathbb E_{x \sim \mathbb P_r}\big[ \ell_r(x,\pi(x)) \big],
\]
for some measurable loss $\ell_r:\mathcal X \times \mathbb R \to \mathbb R$.
Moreover, for $\mathbb P_r$-a.e.\ $x$, the function $a \mapsto \ell_r(x,a)$ has a \emph{unique}
minimizer $a = \pi_r^*(x)$ and is \emph{strictly} larger away from the minimizer:
\[
\ell_r(x,a) > \ell_r(x,\pi_r^*(x)) \quad \text{for all } a \neq \pi_r^*(x).
\]
Finally, the regime distributions overlap on the conflict set in the sense that for any measurable
$B \subseteq \mathcal X_{\mathrm{conflict}}$,
\[
\mathbb P_0(B)>0 \;\Rightarrow\; \mathbb P_1(B)>0,
\qquad
\mathbb P_1(B)>0 \;\Rightarrow\; \mathbb P_0(B)>0.
\]
We also assume that the set of points in the conflict set where the two regime-optimal actions coincide is $\mathbb P_r$-null for each regime:
\[
\mathbb P_r\!\left(\left\{x\in\mathcal X_{\mathrm{conflict}}:\ \pi_0^*(x)=\pi_1^*(x)\right\}\right)=0,
\qquad r\in\{0,1\}.
\]
(This is a mild regularity condition and holds in our continuous simulator unless the two optimizers coincide only on a lower-dimensional crossing set.)
\end{assumption}


\begin{proof}[Proof of \Cref{prop:irreducible_ambiguity}]
Fix any measurable regime-oblivious policy $\pi:\mathcal X \to \mathbb R$. By \Cref{def:semantic_inversion}, there exist a coordinate $x_j$ and a conflict set $\mathcal X_{\mathrm{conflict}} \subset \mathcal X$ such that $\mathbb P_0(\mathcal X_{\mathrm{conflict}}) > 0$ and $\mathbb P_1(\mathcal X_{\mathrm{conflict}}) > 0$, and for all $x \in \mathcal X_{\mathrm{conflict}}$,
\[
\frac{\partial u_0^*(x)}{\partial x_j} \cdot \frac{\partial u_1^*(x)}{\partial x_j} < 0,
\qquad
\min\!\left\{\left|\frac{\partial u_0^*(x)}{\partial x_j}\right|, \left|\frac{\partial u_1^*(x)}{\partial x_j}\right|\right\} \ge \delta,
\]
where $u_r^*(x) = |\pi_r^*(x) - a_{t-1}|$ as in \Cref{def:semantic_inversion}. By \Cref{ass:strict-opt}, the set on which $\pi_0^*(x) = \pi_1^*(x)$ inside $\mathcal X_{\mathrm{conflict}}$ is $\mathbb P_r$-null for each $r$. Hence,
\[
\pi_0^*(x) \neq \pi_1^*(x)
\qquad \text{for } \mathbb P_r\text{-almost every } x \in \mathcal X_{\mathrm{conflict}}, \ r \in \{0,1\}.
\]

Define the disagreement sets
\[
A_r \;:=\; \{x \in \mathcal X_{\mathrm{conflict}} : \pi(x) \neq \pi_r^*(x)\},
\qquad r \in \{0,1\}.
\]
Since $\pi$ is single-valued and $\pi_0^*(x)\neq \pi_1^*(x)$ on $\mathcal X_{\mathrm{conflict}}$,
it is impossible for $\pi(x)$ to equal both $\pi_0^*(x)$ and $\pi_1^*(x)$ at any
$x \in \mathcal X_{\mathrm{conflict}}$. Hence, for every $x \in \mathcal X_{\mathrm{conflict}}$,
we have $x \in A_0 \cup A_1$, i.e.
\[
\mathbb P_r\!\left(\mathcal X_{\mathrm{conflict}}\setminus(A_0\cup A_1)\right)=0,\qquad r\in\{0,1\}.
\]

We now show that at least one disagreement set has positive probability under its corresponding regime. Since $\mathbb P_0(\mathcal X_{\mathrm{conflict}})>0$ and $\mathcal X_{\mathrm{conflict}} \subseteq A_0 \cup A_1$, it follows that either $\mathbb P_0(A_0)>0$ or $\mathbb P_0(A_1)>0$. If $\mathbb P_0(A_0)>0$, we are done by taking $r=0$. Otherwise, $\mathbb P_0(A_1)>0$, and by the overlap condition in \Cref{ass:strict-opt} (with $B=A_1 \subseteq \mathcal X_{\mathrm{conflict}}$), it follows that $\mathbb P_1(A_1)>0$, so we are done by taking $r=1$. Thus, there exists $r\in\{0,1\}$ such that
\[
\mathbb P_r(A_r) > 0.
\]


Fix such an $r$. By \Cref{ass:strict-opt}, for $\mathbb P_r$-almost every $x \in A_r$ we have $\ell_r(x,\pi(x)) > \ell_r(x,\pi_r^*(x))$, while for $x \notin A_r$ we have $\ell_r(x,\pi(x)) \ge \ell_r(x,\pi_r^*(x))$. Therefore,
\[
\mathcal R_r(\pi) - \mathcal R_r(\pi_r^*)
= \mathbb E_{x \sim \mathbb P_r}\!\left[\ell_r(x,\pi(x))-\ell_r(x,\pi_r^*(x))\right]
> 0,
\]
where strict positivity follows because the integrand is strictly positive on a set of positive $\mathbb P_r$-measure, namely $A_r$. This proves that no regime-oblivious policy $\pi$ can be optimal for both regimes simultaneously and establishes the stated strict regret gap in at least one regime.
\end{proof}


%Long table of notations below:
\setlength{\LTleft}{0pt}
\setlength{\LTright}{0pt}

\clearpage
\section{Notation}
\label{app:notation}

\small
\setlength{\tabcolsep}{6pt}
\renewcommand{\arraystretch}{1.15}

\begin{longtable}{@{} l p{0.82\linewidth} @{}}
\caption{Summary of notation used throughout the paper.}
\label{tab:notation}\\
\toprule
Symbol & Meaning \\
\midrule
\endfirsthead

\toprule
Symbol & Meaning \\
\midrule
\endhead

\midrule
\multicolumn{2}{r}{\emph{Continued on next page}}\\
\endfoot

\bottomrule
\endlastfoot

$t=0,1,\ldots,T$ & Discrete hedging times; $T$ is the final time index \\
$S_t$ & Underlying price at time $t$ \\
$K$ & Strike price of the European call liability \\
$L$ & Terminal liability and payoff, $L=\max(S_T-K,0)$ \\
$a_t$ & Hedger’s position (units of underlying) at time $t$ \\
$\Delta a_t$ & Trade at time $t$, $\Delta a_t=a_t-a_{t-1}$, with $a_{-1}=0$ \\
$R\in\{0,1\}$ & Latent regime (drawn once per episode and held fixed) \\
$r$ & Realization and index of the regime (used in subscripts) \\
$\tilde S_t$ & Full market state (including latent regime information) \\
$X_t$ & Regime-oblivious observable state used by the agent \\
$\psi(\cdot)$ & Observation and coarsening map defining $X_t=\psi(\tilde S_t)$ \\
$q_\phi(\cdot\mid X_t)$ & Encoder distribution (representation model), parameterized by $\phi$ \\
$Z_t$ & Latent representation at time $t$, sampled as $Z_t\sim q_\phi(\cdot\mid X_t)$ \\
$\pi_\theta(\cdot)$ & Policy mapping representations to actions, $a_t=\pi_\theta(Z_t)$ \\
$C_{r,t}(\Delta a_t;X_t)$ & Execution cost at time $t$ in regime $r$ \\
$c_{\mathrm{spread}}$ & Proportional spread cost coefficient in \cref{eq:cost} \\
$\lambda_{r,t}(X_t)$ & Regime- and state-dependent marginal impact coefficient in \cref{eq:cost} \\
$\pi_r^*(x)$ & Regime-$r$ optimal action given observable $x$ \\
$\mathcal X$ & Observable state space of $x$ \\
$\mathcal X_{\mathrm{conflict}}$ & Conflict set where the optimal aggressiveness response to a feature flips across regimes (\Cref{def:semantic_inversion}) \\
$\ell_r(x,a)$ & Regime-$r$ one-step (or conditional) loss used in risk definitions \\
$\mathcal R_r(\pi)$ & Expected risk of policy $\pi$ in regime $r$ \\
$Y_r^{\theta,\phi}$ & Terminal hedging error in regime $r$ \cref{eq:terminal_error} \\
$ES_\gamma(\cdot)$ & Expected Shortfall at confidence level $\gamma$ \\
$\mathcal C(\phi)$ & Time-averaged information cost (KL-to-prior) in \cref{eq:info_cost} \\
$\beta$ & Information price / Lagrange multiplier in \cref{eq:occams_objective} \\
$\mathrm{KL}(\cdot\|\cdot)$ & Kullback--Leibler divergence \\
$I(\cdot;\cdot)$ & Mutual information \\
$H(\cdot)$ & Shannon entropy \\
$h(\cdot)$ & Binary entropy function \\
$\Delta_t^{BS},\Gamma_t^{BS}$ & Black--Scholes delta and gamma at time $t$ \\
$\tau_t$ & Time-to-maturity feature at time $t$ \\
$\mathrm{Vol}_t$ & Activity and volume proxy at time $t$ (\Cref{eq:vol_model}) \\
$\mu_v,\sigma_{v,r}$ & Log-mean and regime-dependent log-std for $\mathrm{Vol}_t$ \\
$r_t^{(1)}$ & Most recent observed return, $r_t^{(1)}=\log(S_t/S_{t-1})$ \\
$\hat\sigma_t^{(5)}$ & Rolling realized volatility estimate (e.g.\ 5-step window) \\
$u_t$ & Trading aggressiveness, $u_t=|\Delta a_t|$ \\
$H_q,L_q$ & Top/bottom $q$-quantile time sets of a driver $x_j$ within regime $r$ \\
$W_r$ & Wrong-way trading score in regime $r$ (\Cref{def:wrong_way}) \\
$p(z)$ & Prior over representations in VIB, typically $\mathcal N(0,I)$ \\
$q_\phi(z)$ & Aggregated posterior under the training distribution of $X$ \\
\end{longtable}

\normalsize
\clearpage

\section*{Acknowledgments}
The author thanks colleagues and mentors for helpful discussions and feedback. Any remaining errors are the author’s own.

\begin{thebibliography}{99}

\bibitem[Alemi et al.(2017)]{alemi2017deep}
Alexander~A. Alemi, Ian Fischer, Joshua~V. Dillon, and Kevin Murphy.
\newblock Deep variational information bottleneck.
\newblock \emph{International Conference on Learning Representations (ICLR)}, 2017.

\bibitem[Almgren and Chriss(2001)]{almgren2001optimal}
Robert Almgren and Neil Chriss.
\newblock Optimal execution of portfolio transactions.
\newblock \emph{Journal of Risk}, 3(2):5--39, 2001.

\bibitem[Arjovsky et al.(2019)]{arjovsky2019invariant}
Martin Arjovsky, L\'eon Bottou, Ishaan Gulrajani, and David Lopez-Paz.
\newblock Invariant risk minimization.
\newblock \emph{arXiv preprint arXiv:1907.02893}, 2019.

\bibitem[Black and Scholes(1973)]{blackscholes1973}
Fischer Black and Myron Scholes.
\newblock The pricing of options and corporate liabilities.
\newblock \emph{Journal of Political Economy}, 81(3):637--654, 1973.

\bibitem[Bouchaud et al.(2008)]{bouchaud2008digest}
Jean-Philippe Bouchaud, J.~D. Farmer, and Fabrizio Lillo.
\newblock How markets slowly digest changes in supply and demand.
\newblock In \emph{Handbook of Financial Markets: Dynamics and Evolution}, 2008.

\bibitem[Brunnermeier and Pedersen(2009)]{brunnermeier2009liquidity}
Markus~K. Brunnermeier and Lasse~Heje Pedersen.
\newblock Market liquidity and funding liquidity.
\newblock \emph{The Review of Financial Studies}, 22(6):2201--2238, 2009.

\bibitem[B\"uhler et al.(2019)]{buehler2019deep}
Hans B\"uhler, Lukas Gonon, Josef Teichmann, and Ben Wood.
\newblock Deep hedging.
\newblock \emph{Quantitative Finance}, 19(8):1271--1291, 2019.

\bibitem[B\"uhler et al.(2022)]{buehler2022deepbellman}
Hans B\"uhler, Philipp Murray, and Ben Wood.
\newblock Deep {B}ellman hedging.
\newblock \emph{arXiv preprint arXiv:2207.00932}, 2022.

\bibitem[Cont(2006)]{cont2006model}
Rama Cont.
\newblock Model uncertainty and its impact on the pricing of derivative instruments.
\newblock \emph{Mathematical Finance}, 16(3):519--547, 2006.

\bibitem[Duffie(2020)]{duffie2020intermediation}
Darrell Duffie.
\newblock Intermediation of {U}.{S}. Treasury markets after the Covid-19 crisis.
\newblock \emph{Journal of Economic Perspectives}, 34(4):205--228, 2020.

\bibitem[Gatheral et al.(2012)]{gatheral2012impact}
Jim Gatheral, Thibault Jaisson, and Mathieu Rosenbaum.
\newblock The volatility of high-frequency price changes.
\newblock \emph{Quantitative Finance}, 12(1):47--63, 2012.

\bibitem[Geirhos et al.(2020)]{geirhos2020shortcut}
Robert Geirhos, J{\"o}rn-Henrik Jacobsen, Claudio Michaelis, Richard Zemel,
Wieland Brendel, Matthias Bethge, and Felix~A. Wichmann.
\newblock Shortcut learning in deep neural networks.
\newblock \emph{Nature Machine Intelligence}, 2(11):665--673, 2020.

\bibitem[Glasserman and Xu(2014)]{glasserman2014modelrisk}
Paul Glasserman and Xingbo Xu.
\newblock Robust risk measurement and model risk.
\newblock \emph{Quantitative Finance}, 14(1):29--58, 2014.

\bibitem[Glosten and Milgrom(1985)]{glosten1985bid}
Lawrence~R. Glosten and Paul~R. Milgrom.
\newblock Bid, ask and transaction prices in a specialist market with heterogeneously informed traders.
\newblock \emph{Journal of Financial Economics}, 14(1):71--100, 1985.

\bibitem[Gonon et al.(2021)]{gonon2021deephedging}
Lukas Gonon, Josef Teichmann, and Ben Wood.
\newblock Deep hedging with rough volatility.
\newblock \emph{SIAM Journal on Financial Mathematics}, 12(2):631--663, 2021.

\bibitem[Hansen and Sargent(2008)]{hansen2008robust}
Lars Peter Hansen and Thomas~J. Sargent.
\newblock \emph{Robustness}.
\newblock Princeton University Press, 2008.

\bibitem[Horvath et al.(2021)]{horvath2021deephedging}
Blanka Horvath, Josef Teichmann, and Ben Wood.
\newblock Deep hedging under rough volatility.
\newblock \emph{Quantitative Finance}, 21(8):1351--1376, 2021.

\bibitem[Kyle(1985)]{kyle1985continuous}
Albert~S. Kyle.
\newblock Continuous auctions and insider trading.
\newblock \emph{Econometrica}, 53(6):1315--1335, 1985.

\bibitem[Limmer and Horvath(2024)]{limmer2024robustgans}
Yannick Limmer and Blanka Horvath.
\newblock Robust hedging {GANs}: Towards automated robustification of hedging strategies.
\newblock \emph{Applied Mathematical Finance}, 31(3):164--201, 2024.

\bibitem[Lucas(1976)]{lucas1976}
Robert~E. Lucas, Jr.
\newblock Econometric policy evaluation: A critique.
\newblock \emph{Carnegie-Rochester Conference Series on Public Policy}, 1:19--46, 1976.

\bibitem[Mikkil{\"a} and Kanniainen(2023)]{mikkila2023empirical}
Oskari Mikkil{\"a} and Juho Kanniainen.
\newblock Empirical deep hedging.
\newblock \emph{Quantitative Finance}, 23(1):111--122, 2023.

\bibitem[M{\"u}ller et al.(2024)]{mueller2024fastdeephedging}
Konrad M{\"u}ller, Amira Akkari, Lukas Gonon, and Ben Wood.
\newblock Fast deep hedging with second-order optimization.
\newblock In \emph{ACM International Conference on AI in Finance (ICAIF)}, pages 190--199, 2024.

\bibitem[Murray et al.(2022)]{murray2022continuous}
Phillip Murray, Ben Wood, Hans B{\"u}hler, Magnus Wiese, and Mikko S. Pakkanen.
\newblock Deep hedging: Continuous reinforcement learning for hedging of general portfolios across multiple risk aversions.
\newblock \emph{arXiv preprint arXiv:2207.07467}, 2022.

\bibitem[Rockafellar and Uryasev(2000)]{rockafellar2000optimization}
R.~Tyrrell Rockafellar and Stanislav Uryasev.
\newblock Optimization of conditional value-at-risk.
\newblock \emph{Journal of Risk}, 2(3):21--41, 2000.

\bibitem[Rockafellar and Uryasev(2002)]{rockafellar2002conditional}
R.~Tyrrell Rockafellar and Stanislav Uryasev.
\newblock Conditional value-at-risk for general loss distributions.
\newblock \emph{Journal of Banking \& Finance}, 26(7):1443--1471, 2002.

\bibitem[Sagawa et al.(2020)]{sagawa2020distributionally}
Shiori Sagawa, Pang~Wei Koh, Tatsunori~B. Hashimoto, and Percy Liang.
\newblock Distributionally robust neural networks for group shifts: On the importance of regularization for worst-case generalization.
\newblock \emph{International Conference on Learning Representations (ICLR)}, 2020.

\bibitem[Sims(2003)]{sims2003rational}
Christopher~A. Sims.
\newblock Rational inattention: A research agenda.
\newblock \emph{Deutsche Bundesbank Discussion Paper}, No. 34, 2003.

\bibitem[Tishby et al.(2000)]{tishby2000information}
Naftali Tishby, Fernando~C. Pereira, and William Bialek.
\newblock The information bottleneck method.
\newblock \emph{arXiv preprint physics/0004057}, 2000.

\bibitem[Obizhaeva and Wang(2013)]{obizhaeva2013optimal}
Anna Obizhaeva and Jiang Wang.
\newblock Optimal trading strategy and supply and demand dynamics.
\newblock \emph{Journal of Financial Markets}, 16(1):1--32, 2013.

\bibitem[Avellaneda et al.(1995)]{avellaneda1995pricing}
Marco Avellaneda, Andrew Levy, and Antonio Paras.
\newblock Pricing and hedging derivative securities in markets with uncertain volatilities.
\newblock \emph{Applied Mathematical Finance}, 2(2):73--88, 1995.

\bibitem[Soner et al.(2011)]{soner2011martingale}
H.~Mete Soner, Nizar Touzi, and Jianfeng Zhang.
\newblock Martingale representation theorem for the {G}-expectation.
\newblock \emph{Stochastic Processes and their Applications}, 121(2):265--287, 2011.

\bibitem[Gr{\"u}nwald(2007)]{grunwald2007minimum}
Peter Gr{\"u}nwald.
\newblock \emph{The Minimum Description Length Principle}.
\newblock MIT Press, 2007.

\bibitem[Gulrajani and Lopez-Paz(2021)]{gulrajani2021search}
Ishaan Gulrajani and David Lopez-Paz.
\newblock In search of lost domain generalization.
\newblock \emph{arXiv preprint arXiv:2007.01434}, 2021.

\bibitem[Cover and Thomas(2006)]{cover2006elements}
Thomas~M. Cover and Joy~A. Thomas.
\newblock \emph{Elements of Information Theory}.
\newblock Wiley, 2nd edition, 2006.


\end{thebibliography}

\end{document}